%SI 2014-09-05

\documentclass{article}

\usepackage[T1]{fontenc}
\usepackage[utf8]{inputenc}
\usepackage[swedish]{babel}
\usepackage{fullpage}
\usepackage{amssymb}
\usepackage{bussproofs}
\usepackage{amsmath}
\usepackage{graphicx}
\usepackage{verbatim}
\usepackage{tikz}
\let\emptyset\varnothing


\title{Supplemental Instructions}
\date{
      %Place date here!
     }

\begin{document}
\maketitle

\section*{Repetition}
Dra streck mellan koncept som hänger ihop. \\
\setlength{\tabcolsep}{50pt}
\renewcommand{\arraystretch}{2.0}
\begin{tabular}{ r l }
  $ad-bc$  			 & Determinant av 2x2 matris\\
  $\sqrt{v_{1}^2 + v_{2}^2 + ...}$ & Längd av vektor  \\
  $f_{A}(c\vec{u} + d\vec{v}) = c f_{A}(\vec{u}) + d f_{A}(\vec{v})$ & Linjär avbildning \\
  $B = \big( f(e_x) \quad f(e_y) \quad f(e_z) \big)$   & Basbyte \\
  $||u \times v||$ & Area av parallellogram \\
  $\frac{u \cdot v}{u \cdot u} u$	& Projektion \\
  $\big( e_x \quad e_y \big)$ & Identitetsmatris \\
  $\big( e_y \quad e_x \big)$ & Spegling i y=x \\
  $A^t A x = A^t b$ & Minsta kvadratlösning \\
  \renewcommand{\arraystretch}{1.0}  
  $
  \begin{bmatrix}
    1  &  0  \\
    0  &  -1  
  \end{bmatrix}
  $ & Spegling i y-axeln \\
  
\end{tabular}
\renewcommand{\arraystretch}{1.0}  


\section*{1}
Börja med att skriva om på matrisform.
\\
$R_i$ beteckna rad i.\\
$$
\begin{bmatrix}
    1  & -2   &  a   &  1 \\
    1  & -a   &  2   &  1  \\
    a  & -4   &  4   &  2   \\
\end{bmatrix}
$$

$R_1 * (-1) + R_2 \rightarrow R_2$, \quad
$R_1 * (-a) + R_3 \rightarrow R_3$, \quad

$$
\begin{bmatrix}
    1  & -2     &  a       &  1 \\
    0  & 2-a    &  2-a     &  0  \\
    0  & 2a-4   &  4-a^2   &  2-a   \\
\end{bmatrix}
$$
$a = 2 \implies \infty$ antal lösningar, ty överbestämt matris. \\
$a = -4 \implies $ ingen lösningen. \\
$a \neq 2 $ och $a \neq -4 \implies $ entydlig lösning.


\section*{2}
Skriv om på punkterna, (1, 3/2), (2,8) och (3, 11/2), på följande sätt: \\
\begin{equation}
    \begin{cases}
                   y_1 = kx_1 + l\\
                   y_2 = kx_2 + l \\
                   y_3 = kx_3 + l
    \end{cases}
\end{equation} 
=
\begin{equation}
    \begin{cases}
                   3/2  = k*1 + l\\
                   8    = k*2 + l \\
                   11/2 = k*3 + l
    \end{cases}
\end{equation} 
Gör sedan om till en matris ekvation $Ax = b$ \\
$$
\begin{bmatrix}
    1  &  1 \\
    2  &  1 \\
    3  &  1 \\
\end{bmatrix}
\begin{bmatrix}
    k \\
    l \\
\end{bmatrix}
=
\begin{bmatrix}
    3/2 \\
    8   \\
    11/2  \\
\end{bmatrix}
$$
Vi vill nu lösa $A^t A x = A^t b$. \\
$$
A^t A = 
\begin{bmatrix}
    14  &  6 \\
    6  &  3 
\end{bmatrix}
,\qquad
A^t b = 
\begin{bmatrix}
    34   \\
    15
\end{bmatrix}
$$
Använd gauss för att lösa ekvationen.
$$
\begin{bmatrix}
    14  &  6  & 34 \\
    6   &  3  & 15
\end{bmatrix}
$$
Vilkter leder till att $x = (2,1) \implies k = 2, l = 1$
\section*{3}
$$
\begin{bmatrix}
    1  &  2  &  1  & -1  &  2 \\
    3  &  4  &  5  &  2  &  0 \\
    2  &  2  &  1  &  0  &  2 \\
\end{bmatrix} \rightarrow 
\begin{bmatrix}
    1  &  2  &  1  & -1  &  2 \\
    0  & -2  &  2  &  5  & -6 \\
    0  & -2  & -1  &  2  & -2 \\
\end{bmatrix} \rightarrow
\begin{bmatrix}
    1  &  2  &  1  & -1  &  2 \\
    0  & -2  &  2  &  5  & -6 \\
    0  &  0  & -3  & -3  &  4 \\
\end{bmatrix} \rightarrow
$$

$$
\begin{bmatrix}
    1  &  2  &  1  & -1  &  2 \\
    0  &  1  & -1  &  \frac{-5}{2}  & 3 \\
    0  &  0  &  1  &  1  &  \frac{-4}{3} \\
\end{bmatrix} \rightarrow
\begin{bmatrix}
    1  &  0  &  3  & 4  &  -4 \\
    0  &  1  & -1  &  \frac{-5}{2}  & 3 \\
    0  &  0  &  1  &  1  &  \frac{-4}{3} \\
\end{bmatrix} \rightarrow
\begin{bmatrix}
    1  &  0  &  0  &  1  &  0 \\
    0  &  1  &  0  &  \frac{-3}{2}  & \frac{5}{3} \\
    0  &  0  &  1  &  1  &  \frac{-4}{3} \\
\end{bmatrix}
$$

\noindent
Från detta får vi en lösning med två fria parametrar och alltså geometrisk ett 
plan i $\mathbb{R}^5$. En parametrisering av planet ges av:
$$
\mathbf{x} = s
\begin{bmatrix}
    -1 \\
    3/2 \\
    -1 \\
    1 \\
    0 \\
\end{bmatrix}
+ t
\begin{bmatrix}
    0 \\
    -5/3 \\ 
    4/3 \\
    0 \\
    1 \\
\end{bmatrix}
$$

\section*{4}
Lösning saknas ty pivotelement!

\section*{5}
Tanken är att skriva matrisen på övertriangulär form för att sedan multiplicera diagonalen.
\\
$R_i$ beteckna rad i.\\

$$
\begin{bmatrix}
    1  & 1   &  0   &  1 \\
    2  & 8   &  -1  &  2  \\
    4  & -4  &  0   &  3   \\
    8  &  2  &  1   &  4   \\
\end{bmatrix}
$$

$R_1 * (-2) + R_2 \rightarrow R_2$, \quad
$R_1 * (-4) + R_3 \rightarrow R_3$, \quad
$R_4 * (-8) + R_4 \rightarrow R_4$   


$$
\begin{bmatrix}
    1  & 1   &  0   &  1 \\
    0  & 6   &  -1  &  0  \\
    0  & -8  &  0   &  -1   \\
    0  &  -6  &  1   &  -4   \\
\end{bmatrix}
$$

$R_2 * (8/6) + R_3 \rightarrow R_3$, \quad
$R_2 * (1) + R_4 \rightarrow R_4$   

$$
\begin{bmatrix}
    1  & 1   &  0   &  1 \\
    0  & 6   &  -1  &  0  \\
    0  & 0   &  -4/3   &  -1   \\
    0  & 0   &  0   &  -4   \\
\end{bmatrix}
$$
$1 \cdot 6 \cdot -4/3 \cdot -4 = 32$


\section*{6}
Genom att beräkna determinanten av matrisen med de tre vektorerna som rader 
ser vi att denna är nollskiljd. Detta är fallet {\bf om och endast om} 
vektorerna är linjärt oberoende.

\end{document}
