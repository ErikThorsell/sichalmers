%SI 2014-09-05

\documentclass{article}

\usepackage[T1]{fontenc}
\usepackage[utf8]{inputenc}
\usepackage[swedish]{babel}
\usepackage{fullpage}
\usepackage{amssymb}
\usepackage{bussproofs}

\let\emptyset\varnothing

\title{Supplemental Instructions}
\author{Erik Thorsell \\ 
		\small{erithor@student.chalmers.se}
}
\date{2015-09-17}

\begin{document}
\maketitle

\subsection*{Repetition}
Repetition är moder till all inlärning.
\begin{enumerate}

\item[1.]
En relation $R$ på en mängd $A$ kallas, som bekant:
\begin{itemize}
\item Reflexiv om $\forall x : xRx$
\item Symmetrisk om $\forall x : xRy \Rightarrow yRx$
\item Transitiv om $\forall x : xRy \wedge yRz \Rightarrow xRz$
\end{itemize}
Ge exempel på en relation som är:
\begin{itemize}
\item[a)] En ekvivalensrelation
\item[b)] Transitiv, men inte reflexiv eller symmetrisk
\item[c)] Symmetrisk, men inte reflexiv eller transitiv
\end{itemize}
{\it Tenta 030820}

\item[2.]
En funktion $f : \mathbb{R} \rightarrow \mathbb{R}$ kallas växande om det så fort 
$x < y$ gäller att $f(x) \leq f(y)$. Om det för alla sådana $x$ och $y$ dessutom 
gäller att $f(x) < f(y)$, kallas $f$ {\it strikt/strängt växande}.
\begin{itemize}
\item[a)] Visa att en strikt växande funktion alltid är injektiv.
\item[b)] Ge ett exempel på en funktion som är växande, men inte injektiv.
\end{itemize}
{\it Tenta 050330}

\end{enumerate}

\subsection*{Induktion}
\begin{enumerate}
\item[3.] 
Visa att för alla $n \in \mathbb{Z+}$ gäller att: 
\[\sum\limits_{k=1}^{n} 3k(k-1)+1=n^3\] 
{\it Tenta 021024}

\item[4.]
Visa att det för alla udda positiva heltal $n$ gäller att:
\[1+3+5+7+9+...+n=(\frac{n+1}{2})^2\] 
{\it Tenta 041217}

\item[5.]
Visa att det för alla postiva heltal $n$ gäller att:
\[\sum\limits_{k=1}^{n} \frac{k}{3^k}=\frac{3}{4}-\frac{2n+3}{4*3^n}\]
{\it Tenta 051216}
\end{enumerate}

\subsection*{Rekursion}
\begin{enumerate}

\item[6.]
Fakultet är en funktion inom matematiken. För ett heltal större än noll är fakulteten 
lika med produkten av alla heltal från 1 upp till och med talet självt.\\

Låt $f(x)=x!$ Definiera $f(x)$ rekursivt.

\item[7.]
Fibonaccis talserie är ytterligare ett exempel på något inom matematiken som kan definieras rekursivt.
Visserligen går den även att definiera icke-rekursivt:
\[F(n)=\frac{(1+\sqrt{5})^n-(1-\sqrt{5})^n}{\sqrt{5}*2^n}\]
men hur kul är det? \\

Definiera $F(n)$ rekursivt.

\end{enumerate}

\subsection*{Summor}
\begin{enumerate}
\item[8.]
Beräkna följande summor: 
\begin{itemize}
\item[a)] $\sum\limits_{k=1}^{7} 5^k$
\item[b)] $\frac{1}{3}+\frac{2}{9}+\frac{4}{27}+...+\frac{64}{2187}$
\end{itemize}
\end{enumerate}
\end{doQcument}
