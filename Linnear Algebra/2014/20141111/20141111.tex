%SI 2014-09-05

\documentclass{article}

\usepackage[T1]{fontenc}
\usepackage[utf8]{inputenc}
\usepackage[swedish]{babel}
\usepackage{fullpage}
\usepackage{amssymb}
\usepackage{bussproofs}
\usepackage{amsmath}
\usepackage{graphicx}
\usepackage{verbatim}
\usepackage{tikz}
\let\emptyset\varnothing


\title{Supplemental Instructions}
\author{Benjamin Eriksson \& Erik Thorsell \\ 
		\small{beneri@student.chalmers.se} \&
		\small{erithor@student.chalmers.se}
}
\date{
      %Place date here!
     }

\begin{document}
\maketitle


\section*{Repetition}
\begin{itemize}
	\item[a) ]	Hitta en vektor $u$ som är ortogonal mot vektorn 
				$v = 
				\begin{pmatrix}
					1 \\
					2
				\end{pmatrix}
				$
	\item[b) ]	Finns det några fler vektorer som är ortogonala mot $v$?
\end{itemize}

\subsection*{Projektion och Spegling}
\subsubsection*{1.}
Låt $\vec{u} = (3,1)$ vara riktningsvektorn för linjen L och $\vec{v} = (3,2)$. 

\begin{itemize}
\item[a) ] Hitta den ortogonala projektionen, $\vec{v_L}$ av $\vec{v}$ på L.
\item[b) ] Hitta speglingen, $\vec{v_S}$ av $\vec{v}$ på L.
\end{itemize}

\noindent
\setlength{\unitlength}{0.75mm}
\begin{picture}(80,50)

\multiput(0,0)(5,0){17}%
{\line(0,1){45}}
\multiput(0,0)(0,5){10}%
{\line(1,0){80}}

\put(22,20){$\vec{v}$}
\put(27,11){$\vec{v_{L}}$}
\put(15,8){$\vec{u}$}
\put(60,30){$L$}
\thicklines
\put(10,10){\vector(3,2){30}}
\put(10,10){\vector(3,1){35}}
\put(10,10){\vector(3,1){10}}
\put(0,7){\line(3,1){70}}
\end{picture}
\noindent
\newline
\newline


\subsection*{Linjer och Plan}
\subsubsection*{2.}
Skriv ekvationen för linjen vilken passerar genom punkterna $A=(1,2)$ och 
$B=(2,5)$ på normal form, parameterform och {\it `` y=kx+m-form ''}.

\subsubsection*{3.}
Skriv ekvationen för linjen $r$ vilken passerar genom punkten $A=(1,5)$ och 
är parallell med den räta linjen $s$ mellan punkterna $(4,1)$ och $(-2,2)$.

\subsubsection*{4.}
Ett plan går genom punkterna $A=(1,1,-2)$, $B=(-1,5,2)$ och $C=(3,0,2)$.
Bestäm planets ekvation.\\

\section*{Avstånd}
\subsection*{5.}
\begin{itemize}
    \item[a) ] Beräkna avståndet mellan punkterna $A=(9,2,7)$ \& $B=(4,8,10)$.
    \item[b) ] Beräkna avståndet mellan linjen $-2x+3y+4=0$ och punkten 
               $P=(5,6)$.
    \item[c) ] Beräkna avståndet mellan planet $2x+y-z=-1$ och punkten 
               $P=(3,1,-2)$.
\end{itemize}

\section*{Matriser}
\subsection*{6.}
\begin{itemize}
    \item[a) ]	
    			$ 
    			\begin{bmatrix}
    			1 & 2 \\
    			5 & 9
    			\end{bmatrix}
    			+
    			\begin{bmatrix}
    			-2 & 4 \\
    			6  & 1
    			\end{bmatrix}
    			$
    \item[b) ]	
    			$ 
    			\begin{bmatrix}
    			5 & -7 \\
    			4 & -1
    			\end{bmatrix}
    			-
    			\begin{bmatrix}
    			5   & 5 \\
    			-2  & 8
    			\end{bmatrix}
    			$
    			
    \item[c) ]	
    			$ 
    			\begin{bmatrix}
    			5  & 5 \\
    			2  & 3
    			\end{bmatrix}
    			\cdot
    			\begin{bmatrix}
    			2   \\
    			7  
    			\end{bmatrix}
    			$
    			
    \item[d) ]	
    			$ 
    			\begin{bmatrix}
    			5  & 5 \\
    			2  & 3
    			\end{bmatrix}
    			\cdot
    			\begin{bmatrix}
    			2 & 8  \\
    			7 & 1
    			\end{bmatrix}
    			$
    \item[e) ]	
    			$ 
    			\begin{bmatrix}
    			5 & 7  & 3 \\
    			4 & -6 & 9
    			\end{bmatrix}
    			^{T}
    			$
    		
\end{itemize}

\subsection*{7.}
\begin{itemize}
    \item[a) ]	Beräkna determinanten.
				\newline
				\\
				$ 
    			\begin{vmatrix}
   		 		7  & 4 \\
  		  		1  & 2
		    	\end{vmatrix}
  			  	$
    \item[b) ]	Vad kan sägas om vinkeln mellan vektorerna 
    			$
    			u =   			  
				\begin{pmatrix}
   		 		7  \\
  		  		1  
		    	\end{pmatrix}  		
		    	,
		    	v =   			  
				\begin{pmatrix}
   		 		4  \\
  		  		2  
		    	\end{pmatrix}  			
		    	$ 
		    	utifrån determinanten?
\end{itemize}

\subsection*{8.}
\begin{itemize}
    \item[a) ]	
				$ 
    			\begin{bmatrix}
    			7  & 2 \\
    			3  & 5
    			\end{bmatrix}
    			\cdot
    			\begin{bmatrix}
    			1 & 0  \\
    			0 & 1
    			\end{bmatrix}
    			$
    			
    \item[b) ]	Beräkna inversen 
    			\newline
    			\\
    			$ 
    			\begin{bmatrix}
    			7  & 2 \\
    			3  & 5
    			\end{bmatrix}
				^{-1}
    			$
    			
    \item[c) ]	
    			$ 
    			\begin{bmatrix}
    			7  & 2 \\
    			3  & 5
    			\end{bmatrix}
    			\cdot
    			\begin{bmatrix}
    			7  & 2 \\
    			3  & 5
    			\end{bmatrix}
				^{-1}
    			$
    			
    \item[d) ]	Bevisa att
    			$ 
    			A = 
    			\begin{bmatrix}
    			a  & b \\
    			c  & d
    			\end{bmatrix}  
    			\implies
    			A^{-1} = 
    			\frac{1}{det(A)}
    			\begin{bmatrix}
    			d  & -b \\
    			-c & a
    			\end{bmatrix}		
    			$
    			\\
    			\\
    			\it{Hint: $A A^{-1} = ... $}
\end{itemize}

\end{document}
