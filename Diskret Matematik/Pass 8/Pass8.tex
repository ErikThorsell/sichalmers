%SI 2014-09-05

\documentclass{article}

\usepackage[T1]{fontenc}
\usepackage[utf8]{inputenc}
\usepackage[swedish]{babel}
\usepackage{fullpage}
\usepackage{amssymb}
\usepackage{bussproofs}
\usepackage{amsmath}
\usepackage{graphicx}
\usepackage{verbatim}
\usepackage{tikz}
\let\emptyset\varnothing


\title{Supplemental Instructions}
\author{Erik Thorsell \\ 
		\small{erithor@student.chalmers.se}
}
\date{2015-10-22}

\begin{document}
\maketitle

\section*{Tentakött!}

\begin{enumerate}

\item[1.]
Betrakta mängden $V$ med $n$ element.
\begin{itemize}
    \item[a)] För ett positivt heltal $k < \binom{n}{2}$, hur många grafer med 
              $V$ som nodmängd och med $k$ kanter kan man bilda?\\
              {\it (Kom ihåg att en graf saknar öglor och multipla kanter.)}
    \item[b)] På hur många olika sätt kan man bilda en fullständig bipartit 
              graf med $V$ som nodmängd?
    \item[c)] På hur många sätt kan man bilda en {\it n-väg} med $V$ som 
              nodmängd?
    \item[d)] På hur många sätt kan man bilda en {\it n-cykel} med $V$ som 
              nodmängd?
\end{itemize}
{\it Tentamen 2003-10-23}\\
\\
\item[2.]
    För vilka heltal $n$ gäller att $$21|n^8 - 7n^5 + 6n^3 + 4$$
{\it Tentamen 2003-10-23}\\
\\
\item[3.]
\begin{itemize}
    \item[a)] Antag att $n$ olika bollar ska fördelas på tre urnor på så sätt 
              att i de tre urnorna ska det finnas $k$, $l$, respektive $m$ 
              bollar. På hur många sätt kan det ske?
    \item[b)] Låt $a,b,c \in \mathbb{R}$ och $n \in \mathbb{Z}_{+}$. Som 
              bekant säger binomialsatsen att $$(a+b)^n = \sum_{(k,l):k+l=n} 
              \binom{n}{k}a^kb^l$$
              På liknande sätt gäller att $$(a+b+c)^n = \sum_{(k,l,m):k+l+m=n} 
              A(k,l,m)a^kb^lc^m$$
              Fråga: Vad är $A(k,l,m)?$\\
\end{itemize}
{\it Tentamen 2004-01-16}\\
\\
\item[4.]
    Visa att antalet följder av längd $n$, $n=1,2,3,...$, av nollor och ettor, 
    som innehåller ett udda antal ettor är $2^{n-1}$.

{\it Tentamen 2005-01-11}\\
\\
\item[5.]
    Följden $f_1, f_2, f_3,...$ är given av att $f_1=1$ och att det för $n \ge 
    2$ gäller att $f_n = \sqrt{1+f^2_{n-1}}$.\\
    Visa att det för alla $n$ gäller att $f_n = \sqrt{n}$.\\
{\it Tentamen 2004-12-17}\\

\item[6.]
    Låt $f_n$ vara antalet följder av längd $n$ av talen 1, 2 och 3 sådana att 
    det aldrig står en etta omedelbart före en tvåa.\\
    Finn en rekursiv formel för följden $f_1, f_2, f_3,...$.\\
{\it Tentamen 2004-12-17}\\

\item[7.]
    Låt $G=(V,E)$ vara en graf och låt $R$, $S$ och $T$ vara relationer på $V$ 
    givna att:
    \begin{itemize}
        \item[$\bullet$] $xRy$ om $x$ och $y$ är grannar.
        \item[$\bullet$] $xSy$ om $x$ och $y$ ligger i samma sammanhängande 
                       komponent.
        \item[$\bullet$] $xTy$ om $d_{x} \leq d_{y}$.
    \end{itemize}
    Vilka av relationerna $R$, $S$ och $T$ är ekvivalensrelationer?\\
{\it Tentamen 2005-08-16}\\
\\
\item[8.]
    Bevisa att man kan dela med noll.


\end{enumerate}
\end{document}
