%SI 2014-09-05

\documentclass{article}

\usepackage[T1]{fontenc}
\usepackage[utf8]{inputenc}
\usepackage[swedish]{babel}
\usepackage{fullpage}
\usepackage{amssymb}
\usepackage{bussproofs}
\usepackage{amsmath}
\usepackage{wrapfig}
\usepackage{float}
\usepackage{graphicx}
\usepackage{verbatim}
\usepackage{tikz}
\let\emptyset\varnothing


\title{Supplemental Instructions}
\date{
      %Place date here!
     }

\begin{document}
\maketitle


\section*{1}
Separabel\\
$\frac{dy}{dx} = \frac{x}{y}$\\
$y dy = x dx$\\
$\frac{y^2}{2} = \frac{x^2}{2} + C$\\\\
$y(1) = 2 \Rightarrow 2 = \frac{1}{2} + C \Rightarrow C = \frac{3}{2}$\\\\
$y^2 = x^2+3$\\
$y = \sqrt{x^2+3}$\\

\section*{2}
$y' - y = 2xe^x$\\
Integrerande faktor: $e^{-x}$\\\\
$e^{-x}y' - e^{-x}y=2x$\\
$(e^{-x}y)' = 2x$\\
$e^{-x}y = x^2+C$\\\\
$y(0) = 1 \Rightarrow C = 1$\\
$y = x^2e^x+e^x$\\

\section*{3}
$y''+y = x$\\
Kar. ekv: $r^2+1 = 0 \Rightarrow r = \pm i$\\\\
Homogenlösning: $y = A*cos(x) + B*sin(x)$\\
Partikulärlösning: $y = x$, $y''=0$\\\\
$y = x + A*cos(x) + B*sin(x)$\\
$y' = 1 - A*sin(x) + B*cos(x)$\\\\
$y(0) = y'(0) = 0 \Rightarrow A = 0, 1-B = 0$\\\\
$y = x + sin(x)$\\

\section*{4}
$y''-3y'+2y = sin(2x)$\\
Kar. ekv: $r^2-3r+2=0 \Rightarrow r_1 = 2, r_2 = 1$\\\\
Homogenlösning: $y = Ae^{2x} + Be^x$\\
Partikulärlösning: \\
$y = Acos(2x) + Bsin(2x)$\\
$y' = -2Asin(2x) + 2Bcos(2x)$\\
$y'' = -4Acos(2x) - 4Bsin(2x)$\\\\
$y''-3y'+2y = cos(2x)(-4A - 6B + 2A) + sin(2x)(-4B + 6A + 2B)$\\\\
$\left\{
    \begin{array}{l}
        -2A - 6B = 0\\
        -2B + 6A = 1
    \end{array}\right.$\\\\
$A = -3B$\\
$-B - 9B = \frac{1}{2}$\\
$B = -\frac{1}{20}$\\
$A = \frac{3}{20}$\\\\
Lösning:\\
$y = \frac{3}{20}cos(2x) - \frac{1}{20}sin(2x) + A e^{2x}+Be^x$\\
$y' = -\frac{3}{10}sin(2x) - \frac{1}{10} cos(2x) + 2Ae^{2x} + Be^x$\\\\
$y(0) = y'(0) \Rightarrow$\\
$\frac{3}{20} + A + B = 0$\\
$- \frac{1}{10} + 2A + B = 0$\\
$A = \frac{5}{20} = \frac{1}{4}$\\
$B = - \frac{8}{20} = - \frac{2}{5}$\\\\
Svar: $y=\frac{3}{20}cos(2x)-\frac{1}{20}sin(2x)+\frac{1}{4}e^{2x}-\frac{2}{5}e^x$

\section*{5}
Skivformeln ger:\\
$\int_{0}^{2} \pi (e^x\sqrt{x})^2 dx = $\\
$\pi \int_0^2 x*e^{2x}dx = $\\
$\pi [\frac{x*e^{2x}}{2}]_0^2 - \frac{\pi}{2} \int_0^2 e^{2x}dx = $\\
$\pi((e^4-0) + (\frac{1}{4} - \frac{e^4}{4})) = $\\
$\pi(\frac{1}{4} + e^4 - \frac{e^4}{4}) = $\\
$\pi(\frac{1}{4} + \frac{3e^4}{4})$

\section*{6}
Om vi vrider $8-6i$ $90^{\circ}$ får vi $i(8-6i) = 8i+6 = 2(3+4i)$ vilket är 
parallellt med $(3+4i)$.\\\\
\noindent
Alternativt kan vi använda omvändningen av Pythagoras sats:\\
$\text{Hypotenusan}^2 = |8-6i - (3+4i)|^2 = |5-10i|^2 = 125 = $\\
$(64+36) + (9+16) = |8-6i|^2 + |3+4i|^2$

\section*{7}
Om $v(t)$ är hastigheten har vi $v' = -k \sqrt(v)$, $v(0) = v_0$.\\
$\frac{dv}{\sqrt{v}} = -k dt$\\
$2 \sqrt{v} = -kt + C$\\\\
$t=0 \text{ ger } C = 2\sqrt{v_0}$\\
$2\sqrt{v} = 2\sqrt{v_0} - kt = 0$\\
för $ t = \frac{2\sqrt{v_0}}{k}$\\

\section*{8}
Eulers metod är: $y_{k+1} - y_k = h({x_k}^2 + {y_k}^2)$\\\\
$y(0) = 0$\\
$y(\frac{1}{2}) = 0 + \frac{1}{2}(0^2 + 0^2) = 0$\\
$y(1) = 0 + \frac{1}{2}({\frac{1}{2}}^2 + 0^2) = \frac{1}{8}$\\\\
Svar: $0, 0, \frac{1}{8}$


%UPPGIFT 9%
\subsection*{9}
$y^2 = 1 + \int_{1}^{x} {y^3 dt}$ \\
Tricket här är analysens andra fundamentalsats: 
$\frac{d}{dx} \int_{c}^{x} {f(t) dt} = f(x)$ \\\\
Derivera båda sidorna:
$2y(x) y' = y(x)^3$ \\
Nu behöver vi separera: $2y(x) \frac{dy}{dx} = y(x)^3$ \\ 
$\frac{1}{y(x)^{2}} \frac{dy}{dx} = \frac{1}{2}$ \\ 
$\frac{1}{y(x)^{2}} \> dy = \frac{1}{2} \> dx$ \\ 
$-\frac{1}{y} = \frac{x}{2} + c$ \\
$\frac{1}{y} = -\frac{x+c}{2} $ \\
$y = -\frac{2}{x+c} $ \\\\
C kan vi beräkna från orginalekvationen. Vi väljer något x så att integralen blir enkel. I det här fallet passar x=1 bra. \\ 
$(-\frac{2}{1+c})^2 = 1 + \int_{1}^{1} {y^3 dt}$ \\\\
$\frac{4}{(1+c)^2} = 1 + 0$ \\
Vi ser här att c=1 uppfyller kravet. \\
$y = -\frac{2}{x+1} $ \\

\end{document}
