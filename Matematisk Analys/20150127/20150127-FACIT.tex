%SI 2014-09-05

\documentclass{article}

\usepackage[T1]{fontenc}
\usepackage[utf8]{inputenc}
\usepackage[swedish]{babel}
\usepackage{fullpage}
\usepackage{amssymb}
\usepackage{bussproofs}
\usepackage{amsmath}
\usepackage{graphicx}
\usepackage{verbatim}
\usepackage{tikz}
\let\emptyset\varnothing


\title{Supplemental Instructions}
\date{
      %Place date here!
     }

\begin{document}
\maketitle

\section*{1}
\begin{itemize}
	\item[a) ]	Definitionsmängd: $x \in \mathbb{R}, x \neq -4$ \\
				Målmängd: $x \in \mathbb{R}, x \neq 0$
				
	\item[b) ] 	$x + 4$			
\end{itemize}
\section*{2}
$sin^4 x - cos^4 x = 2 \> sin^2 x - 1$	\\ \\
$(sin^2 x - cos^2 x)(sin^2 x + cos^2 x) = 2 \> sin^2 x - 1$ \\ \\
$(sin^2 x - cos^2 x) * 1 = 2 \> sin^2 x - 1$ \\ \\
$(sin^2 x - (1-sin^2 x)) * 1 = 2 \> sin^2 x - 1$ \\ \\
$2 \> sin^2 x - 1 = 2 \> sin^2 x - 1$ \\ 
\section*{3}
$\lim_{x \to (-a)} \quad \frac{x^2 - a^2}{x + a}$ \\ \\
$\frac{x^2 - a^2}{x + a} = \frac{(x - a)(x + a)}{x + a} = x - a$ \\ 
$\lim_{x \to (-a)} \quad \frac{x^2 - a^2}{x + a} =
\lim_{x \to (-a)} \quad x - a = (-a) - a = -2a$ 

\section*{4}
\begin{itemize}
\item[a )]
1) All polynomials are continuous \\
2) The intermediate value theorem. 
\item[b )]
Att högra och vänstra gränsvärdet finns men stämmer ej överense med funktionen.
\\
Alltså: $\lim_{x \to c} f(x)$ är väldefininerad men
$\lim_{x \to c} f(x) \neq f(c)$.
\end{itemize}



\section*{5}
\begin{itemize}
    \item[a) ] Nej.
    \item[b) ] Lutningen för $f(x) = |x^2 - 1|$ i $x = 0$ är
               $$m = \lim_{h \to \infty} \frac{|(1+h)^2 - 1| - |1-1|}{h}
                   = \lim_{h \to \infty} \frac{|2h+h^2|}{h}$$
               Gränsvärdet existerar ej och är varken $- \infty$ eller 
               $\infty$. Grafen har ingen tangent i $x = 1$.
    \item[c) ] Lutningen för $f(x) = (x-1)^{4/3}$ i $x = 0$ är
               $$m = \lim_{h \to \infty} \frac{(1+h-1)^{4/3}-0}{h}
                   = \lim_{h \to \infty} h^{1/3} = 0$$
               Grafen har en tangent med lutning $0$ i $x = 1$. 
               Eftersom $f(1) = 0$ är tangentens ekvation $y = 0$.
\end{itemize}

\section*{6}
\begin{itemize}
    \item[a) ] $y' = 2x - 3$
    \item[b) ] $y' = - \frac{4}{(x+2)^2}$
    \item[c) ] $y' = - \frac{2}{t^3}$
\end{itemize}

\section*{7}
\begin{enumerate}
    \item Korrekt.
    \item Inkorrekt! $(fg)'(x) = f'(x) g(x) + f(x) g'(x)$
    \item Inkorrekt! Om $f$ är deriverbar i $x$ och $f(x) \neq 0$ stämmer dock 
          påståendet.
    \item {\it Med ovanstående ändring.} Korrekt.
    \item Korrekt.
    \item Inkorrekt. Detta är kedjeregeln. 
          {\it Leta nu upp distributiva deriveringsregeln!}
\end{enumerate}
\begin{itemize}
    \item[a) ] $f' = 13 - 30s$
    \item[b) ] $f' = \frac{\pi ^2}{(2-\pi p)^2}$
    \item[c) ] $g' = - \frac{24}{(4v + 3)^2}$
    \item[d) ] $y' = - \frac{3x}{\sqrt{1-3x^2}}$
    \item[e) ] $-2 sin(x) cos(2 cos(x))$
    \item[f) ] $2(sin(2x) + cos(2x))$
    \item[g) ] $sec^2(x) sin(tan(x)) (-cos(cos(tan(x))))$
\end{itemize}

\section*{8}
\begin{itemize}
    \item[a) ] $y' = - 14(3-2x)^6$ \\ $y'' = 168(3-2x)^5$ \\ $y''' = -1680(3-2x)^4$
    \item[b) ] $y' = \frac{x cos(x) - sin(x)}{x^2}$ \\
               $y'' = \frac{2(x(cos(x) - sin(x))}{x^3} - \frac{sin(x)}{x}$ \\
               $y''' = \frac{6(x cos(x) - sin(x))}{x^4} + \frac{3 sin(x)}{x^2}- \frac{cos (x)}{x}$
    \item[c) ] $y' = \frac{5 x^2 + 3}{2 \sqrt{x}}$ \\
               $y'' = \frac{3(5x^2-1)}{4x^{3/2}}$ \\
               $y''' = \frac{3(5x^2+3}{8x^{5/2}}$
\end{itemize}


\end{document}
