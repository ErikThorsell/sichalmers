%SI 2014-09-05

\documentclass{article}

\usepackage[T1]{fontenc}
\usepackage[utf8]{inputenc}
\usepackage[swedish]{babel}
\usepackage{fullpage}
\usepackage{amssymb}
\usepackage{bussproofs}
\usepackage{amsmath}
\usepackage{graphicx}
\usepackage{verbatim}
\usepackage{tikz}
\let\emptyset\varnothing


\title{Supplemental Instructions}
\author{Erik Thorsell \\ 
		\small{erithor@student.chalmers.se}
}
\date{2015-10-22}

\begin{document}
\maketitle

\section*{Tentakött!}

\begin{enumerate}

\item[1.]
\begin{itemize}
    \item[a)] Antalet grafer sammanfaller med antalet sätt att välja ut $k$ 
              stycken bland alla de $\binom{n}{2}$ paren av element man kan
              bilda från V. Svaret blir således: $$\binom{\binom{n}{2}}{k}$$
    \item[b)] Här handlar det om antalet sätt de n noderna kan delas in i två
              grupper. Låt oss kalla dem ``de röda noderna'' och ``de blå
              noderna''. För varje nod finns det två alternativ, blå eller röd,
              så enligt multiplikationsprincipen finns det $2^n$ olika sätt att
              dela in noderna i röda och blå noder. Dock gäller det att varje
              fullständig bipartit graf kan uppkomma från precis två sådana
              uppdelningar, vilka fås från varandra genom att kalla alla blå
              noder röda och vice versa. Således är antalet fullständiga -
              bipartita - grafer $2^{n-1}$.
    \item[c)] Varje n-väg svarar mot precis en permutation av de n noderna
              varför det finns precis $n!$ stycken n-vägar.
    \item[d)] Varje n-väg svarar mot precis n stycken n-väga (ty det finns
              precis n ställen där en cykel kan ``klippas av'' och bilda en
              väg), varför svaret blir: $$\frac{n!}{n} = (n-1)!$$\\
\end{itemize}

\item[2.]
    Kalla högerledet $f(n)$, dvs $f(n)=n^8+6n^3+4$. Eftersom 21 kan skrivas som
    produkten av två primtal (7 och 3) gäller att $21|f(n)$ om och endast om
    $3|f(n)$ och $7|f(n)$. Detta kan skrivas om som: $f(n) \equiv 0 \text{
    modulo } 3$ och $f(n) \equiv 0 \text{ modulo } 7$. Vid räkning i 
    $\mathbb{Z}_7$ gäller att $f(n) = n^8+6n^3+4 = n^2-n^3+4$.
    $$f(0) = 4$$
    $$f(1) = 4$$
    $$f(2) = 0$$
    $$f(3) = 0$$
    $$f(4) = 4$$
    $$f(5) = 2$$
    $$f(6) = 6$$
    Vi ser att att $f(n) \equiv 0 \text{ mod } 7$ omm $n \equiv 2$ eller $n \equiv 3
    \text{ mod } 7$. Vid räkning i $\mathbb{Z}_3$ gäller att $f(n)=1-n^5+4=5-n-2=2-n$
    varför $f(n) \equiv 0 \text{ mod } 3$ omm $n \equiv 2 \text{ mod } 3$.\\
    Summa summarum gäller alltså att $21|f(n)$ gäller då:
    \[\left\{
        \begin{array}{lr}
            n \equiv 2 \text{ mod } 3\\
            n \equiv 2 \text{ mod } 7
        \end{array}
    \right.\]
    eller
    \[\left\{
        \begin{array}{lr}
            n \equiv 2 \text{ mod } 3\\
            n \equiv 3 \text{ mod } 7
        \end{array}
    \right.\]

    En lösning till ekvationssytemet är att $n=2$ så enligt kinesiska
    restsatsen gäller att den allmänna lösningen är $n=2+21m$, $m \in
    \mathbb{Z}$. Till det andra ekvationssystemet är $n=17$ en lösning så den
    allmänna lösningen är $n=17+21m$, $m \in \mathbb{Z}$. De $n$ vi söker är
    alltså de $n$ som kan skrivas som $2+21m$ eller $17+21m$ där m är något
    heltal.\\

\item[3.]
\begin{itemize}
    \item[a)] Först väljer vi ut bollarna till urna 1. Detta kan ske på
              $\binom{n}{k}$ olika sätt. Bland de återstående $n-k$ bollarna
              väljs sedan $l$ bollar ut till urna 2. Detta kan ske på
              $\binom{n-k}{l}$ olika sätt. Till sist finns det bara ett val;
              att lägga de återstående $n-k-l=m$ bollarna i urna 3. Enligt
              multiplikationsprincipen finns totalt: $$\binom{n}{k}
              \binom{n-k}{l} = \frac{n!}{k! l! m!}$$ sätt att placera ut
              bollarna.
    \item[b)] $A(k,l,m)$ är antalet sätt att välja ut $k$ parenteser att plocka
              $a$ ur, $l$ parenteser att plocka $b$ ur och $m$ parentser att
              plocka $c$ ur. Genom att associera $a$ med urna 1, $b$ med urna 2
              och $c$ med urna 3 i uppgift (a) kan man efter lite
              tankegymnastik komma fram till att $A(k,l,m)$ blir just svaret i
              (a).\\
\end{itemize}

\item[4.]
    Använd induktion. Uppgiftens påstående är uppenbart sant då $n=1$ så antag
    att det gäller även för $n=m$ där $m$ är ett godtyckligt valt positivt
    heltal. Antagandet säger att precis hälften av följderna av längd $m$
    innehåller ett udda antal ettor. Därför måste också precis hälften av
    följderna innehålla ett jämt antal ettor. Nu är ju antalet följder av läng
    $m+1$ med ett udda antal ettor dels de som börjar med en etta och följd av
    en följd mav längd $m$ med ett jämt antal ettor, dels de som börjar med en
    nolla och följs av en följd av längd $m$ med ett udda antal ettor. Summerar
    vi dessa två antal får vi $2^{m-1}+2^{m-1} = 2^m$ och det önskade
    resultatet följer av induktionsprincipen.\\\\

\item[5.]
    $f_1=1=\sqrt{1}$ så om vi under antagandet $f_n = \sqrt{n}$ för ett givet
    $n$ kan visa att $f_{n+1}=\sqrt{n+1}$ följer det önskade resultatet av
    induktionsprincipen. Men detta är rättframt: $$f_{n+1} = \sqrt{1+f_{n}^2} =
    \sqrt{1+\sqrt{n}^2} = \sqrt{n+1}$$\\

\item[6.]
    Uppenbarligen är $f_1=3$. För att bilda en rekursiv formel, antag att vi
    känner till $f_{n-1}$ för ett givet $n$. Av varje godkänd följd av läng
    $n-1$ kan man bilda en följd av längd $n$ genom att sätta en av siffrorna
    1, 2, eller 3 omedelbart före följden. På detta sätt kan man bilda
    $3f_{n-1}$ olika följder.\\
    Nu är inte alla dessa godkända, nämligen de följder som man får genom att
    ta en ``längd-$n-1$-följd'' som börjar med en tvåa och sätta en etta
    framför. Antalet sådana är $f_{n-2}$ ty med en tvåa i början bildar man en
    godkänd följd genom att sätta vilken godkänd följd som helst efter denna
    tvåa. Den korrekta rekursiva formeln blir alltså:
    $$f_{n-3}-f_{n-1}-f_{n-2}$$ med startvärdet $f_1=3$.\\

\item[7.]
    Relationen är $R$ är ingen ekvivalensrelation, ty den är inte transitiv.
    Att $x$ är granne med $y$ och $y$ är granne med $z$ garanterar inte att $x$
    och $z$ är grannar.\\
    Inte heller $T$ är en ekvivalensrelation, den är till exempel inte
    symmetrisk. Om $d_x=1$ och $d_y=2$ gäller ju att $xTy$ men inte
    $yTx$.\\
    Relationen $S$ är däremot en ekvivalensrelation. Varje nod är naturligtvis
    samma komponent som sig själv, och om det finns en väg från $x$ till $y$ så
    finns det en väg från $y$ till $x$. Den är transitiv för att om det finns
    en väg från $x$ till $y$ och en väg från $y$ till $z$ så finns där en väg
    från $x$ till $z$ via $y$.\\

\end{enumerate}
\end{document}
