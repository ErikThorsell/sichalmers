%SI 2014-09-05

\documentclass{article}

\usepackage[T1]{fontenc}
\usepackage[utf8]{inputenc}
\usepackage[swedish]{babel}
\usepackage{fullpage}
\usepackage{amssymb}
\usepackage{bussproofs}
\usepackage{amsmath}
\usepackage{graphicx}
\usepackage{verbatim}
\usepackage{tikz}
\let\emptyset\varnothing


\title{Supplemental Instructions}
\author{Benjamin Eriksson \& Erik Thorsell \\ 
		\small{beneri@student.chalmers.se} \&
		\small{erithor@student.chalmers.se}
}
\date{
      %Place date here!
     }

\begin{document}
\maketitle


\section*{Repetition}
Dra streck mellan koncept som hänger ihop. \\
\setlength{\tabcolsep}{50pt}
\renewcommand{\arraystretch}{2.0}
\begin{tabular}{ r l }
  $ad-bc$  			 						& Spegling i y=x\\
  $\sqrt{v_{1}^2 + v_{2}^2 + ...}$ 			& Projektion  \\
  $f_{A}(c\vec{u} + d\vec{v}) = c f_{A}(\vec{u}) + d f_{A}(\vec{v})$ 	& Determinant av 2x2 matris \\
  $B = \big( f(e_x) \quad f(e_y) \quad f(e_z) \big)$   					& Identitetsmatris \\
  $||u \times v||$ 							& Spegling i y-axeln \\
  $\frac{u \cdot v}{u \cdot u} u$			& Längd av vektor \\
  $\big( e_x \quad e_y \big)$ 				& Minsta kvadratlösning \\
  $\big( e_y \quad e_x \big)$ 				& Linjär avbildning \\
  $A^t A x = A^t b$ 						& Basbyte \\
  \renewcommand{\arraystretch}{1.0}  
  $
  \begin{bmatrix}
    1  &  0  \\
    0  &  -1  
  \end{bmatrix}
  $ 										& Area av parallellogram \\
  
\end{tabular}
\renewcommand{\arraystretch}{1.0}  

\section*{Linjära ekvationssystem}

\subsection*{1}
Avgör för varje värde på parametern $a$ hur många lösningar ekvationssystemet
\begin{equation}
    \begin{cases}
                    x-2y+az = 1 \\
                    x-ay+2x = 1 \\
                    ax - 4y + 4z = 2
    \end{cases}
\end{equation} 
har. Bestäm alla lösningar i de fall då ekvationssystemet inte har entydlig lösning. (Tentamen 2013-04-05)

\subsection*{2}
Bestäm den räta linjen som i minsta-kvadratmening är bäst anpassad till punkterna: (1, 3/2), (2,8) och (3, 11/2).

\subsection*{3}
Motivera att alla lösningar till ekvationssystemet $A \mathbf{x} = 0$ där 
$$ A = 
\begin{bmatrix}
    1  &  2  &  1  & -1  &  2 \\
    3  &  4  &  5  &  2  &  0 \\
    2  &  2  &  1  &  0  &  2 \\
\end{bmatrix}
$$
utgör ett plan i $\mathbb{R}^5$ och bestämt en ekvation på parameterform för 
detta plan.

\subsection*{4}
Bestäm alla lösningar till ekvationssytemet $A \mathbf{x}= \mathbf{b}$ för 
följande totalmatris.
$$ T = 
\begin{bmatrix}
    5  &  7  &  1  &  3  &  2 \\
    3  & -4  &  5  &  2  &  0 \\
    2  &  3  &  0  &  3  &  2 \\
    0  &  0  &  0  &  0  &  3 \\
\end{bmatrix}
$$

\subsection*{5}
Bestäm determinanten till följande matris.
$$
\begin{bmatrix}
    1  & 1   &  0   &  1 \\
    2  & 8   &  -1  &  2  \\
    4  & -4  &  0   &  3   \\
    8  &  2  &  1   &  4   \\
\end{bmatrix}
$$


\subsection*{6}
Är de tre vektorerna
$$
\begin{bmatrix}
    1 \\
    2 \\
    2 \\
\end{bmatrix}
,
\begin{bmatrix}
    2 \\
    2 \\
    5 \\
\end{bmatrix}
och
\begin{bmatrix}
    -1 \\
    -1 \\
     2 \\
\end{bmatrix}
$$
linjärt oberoende? Utgör de en bas för $\mathbb{R}^3$? {\bf Motivera!}

\end{document}
