%SI 2014-09-05

\documentclass{article}

\usepackage[T1]{fontenc}
\usepackage[utf8]{inputenc}
\usepackage[swedish]{babel}
\usepackage{fullpage}
\usepackage{amssymb}
\usepackage{bussproofs}
\usepackage{amsmath}
\usepackage{graphicx}
\usepackage{verbatim}
\usepackage{tikz}
\let\emptyset\varnothing


\title{Supplemental Instructions}
\date{
      %Place date here!
     }

\begin{document}
\maketitle

\section*{Repetition}
Normalen till planet ges av $\overrightarrow{n}=\overrightarrow{AB} \times 
\overrightarrow{AC} = (20,16,-6)$\\
Vi kan sedan använda punkten $A$ och vektorn $\overrightarrow{n_{2}}=
(10,8,-3)$ som är parallell med $\overrightarrow{n}$.\\
$A(x-x_{1})+B(y-y_{1})+C(z-z_{1})=0 \Rightarrow \\
10(x-1)+8(y-1)-3(z+2)=0 \Rightarrow \\
10x+8y-3z-24=0$

\section*{1}
\subsection*{a)}
Addera cellvis.
$ 
    			\begin{bmatrix}
    			1-2 & 2+4 \\
    			5+6 & 9+1
    			\end{bmatrix}
    			=
    			\begin{bmatrix}
    			-1 & 6 \\
    			11  & 10
    			\end{bmatrix}
    			$
    			
\subsection*{b)}
Subtrahera cellvis.
$ 
    			\begin{bmatrix}
    			5-5 & -7-5 \\
    			4-(-2) & -1-8
    			\end{bmatrix}
    			=
    			\begin{bmatrix}
    			0 & -12 \\
    			6  & -9
    			\end{bmatrix}
    			$
    			
\subsection*{c)}
Tänk rad gånger kolumn.
$ 
    			\begin{bmatrix}
    			2*5+5*7  \\
    			2*2+3*7
    			\end{bmatrix}
    			=
    			\begin{bmatrix}
    			45  \\
    			25
    			\end{bmatrix}
    			$

\subsection*{d)}

$ 
    			\begin{bmatrix}
    			2*5+5*7 & 5*8+5*1  \\
    			2*2+3*7 & 2*8+3*1
    			\end{bmatrix}
    			=
    			\begin{bmatrix}
    			45 & 45\\
    			25 & 19
    			\end{bmatrix}
    			$

\subsection*{e)}
Här byter vi plats på raderna och kolumnerna.
				$ 
    			\begin{bmatrix}
    			5 & 4 \\
    			7 & -6 \\
    			3 & 9 
    			\end{bmatrix}
    			$
    			
    			
\section*{2}
\subsection*{a)}
				$ 
    			\begin{vmatrix}
   		 		7  & 4 \\
  		  		1  & 2
		    	\end{vmatrix}
		    	=
		    	7*2 - 1*4
		    	= 10
  			  	$
\subsection*{b)}
$Det(A) \neq 0 \implies \vec{u}$ och $\vec{v}$ är linjärt oberoende. Vilket betyder att vinkeln är skild från 0 och 180.

\section*{3}
\subsection*{a) }
	$ 
    			\begin{bmatrix}
    			7  & 2 \\
    			3  & 5
    			\end{bmatrix}
    			\cdot
    			\begin{bmatrix}
    			1 & 0  \\
    			0 & 1
    			\end{bmatrix}
    			=
    			\begin{bmatrix}
    			7  & 2 \\
    			3  & 5
    			\end{bmatrix}
    			$
    			
\subsection*{b) }
				$
				\begin{bmatrix}
    			7  & 2 \\
    			3  & 5
    			\end{bmatrix}
				^{-1}
				=
				\frac{1}{7*5-2*3}
				\begin{bmatrix}
    			5  & -2 \\
    			-3  & 7
    			\end{bmatrix}
				$
\subsection*{c) }
$				A A^{-1} = I =
				\begin{bmatrix}
    			1  & 0 \\
    			0  & 1
    			\end{bmatrix}
    			$ 
\subsection*{d) }
				$
				\begin{bmatrix}
    			a  & b \\
    			c  & d
    			\end{bmatrix}  
    			\cdot
    			\frac{1}{det(A)}
    			\begin{bmatrix}
    			d  & -b \\
    			-c & a
    			\end{bmatrix}
    			=
    			\frac{1}{det(A)}
    			\begin{bmatrix}
    			ad - bc  & -ab + ab \\
    			cd - cd  & -bc + ad
    			\end{bmatrix} 
    			=
    			\frac{1}{ad-bc}
    			\begin{bmatrix}
    			ad - bc  & -ab + ab \\
    			cd - cd  & -bc + ad
    			\end{bmatrix} 
    			=
    			\begin{bmatrix}
    			1 & 0 \\
    			0 & 1
    			\end{bmatrix} 
    			$

\section*{4}

	\subsection*{a) }	$f_{D}(\vec{u}) 
				=
				\begin{bmatrix}
    			5	&	2	&	-1	\\
    			6	&	3	&	7	\\
    			-3	&	2	&	3
    			\end{bmatrix}   
    			\cdot
    			\begin{bmatrix}
    			2   	\\
   				-1 		\\
   				3		
   				\end{bmatrix}
   				=
				\begin{bmatrix}
    			5 \\
    			30 \\
    			1
    			\end{bmatrix} 				
				$
				
	\subsection*{b) }	$f_{D}(\vec{v}) 
				=
				\begin{bmatrix}
    			5	&	2	&	-1	\\
    			6	&	3	&	7	\\
    			-3	&	2	&	3
    			\end{bmatrix}   
    			\cdot
    			\begin{bmatrix}
    			4   	\\
   				3 		\\
   				-1		
   				\end{bmatrix}
   				=
				\begin{bmatrix}
    			27   	\\
   				26		\\
   				-9		
   				\end{bmatrix} 				
				$
	\subsection*{c) }	$f_{D}(\vec{u} + \vec{v})
				=
				f_{D}(\vec{u}) + f_{D}(\vec{v})	=
				\begin{bmatrix}
    			32   	\\
   				56		\\
   				-8		
   				\end{bmatrix} 	
   				$
	\subsection*{d) }	$f_{D}(\vec{2u})
				=
				2f_{D}(\vec{u})
				=
				\begin{bmatrix}
    			10 \\
    			60 \\
    			2
    			\end{bmatrix}
				$


\section*{5}
Dela upp i enhetsvektorer.
\\
$
	(1) \qquad
	f \bigg (  
    \begin{pmatrix}
    2		\\
    1		\\
    \end{pmatrix}  
	\bigg )   
	=
	2 f \bigg (  
    \begin{pmatrix}
    1		\\
    0		\\
    \end{pmatrix}  
	\bigg )  
	+
	f \bigg (  
    \begin{pmatrix}
    0		\\
    1		\\
    \end{pmatrix}  
	\bigg ) 
	=
	2 f_x + f_y	
	=
	\begin{pmatrix}
    8		\\
    10		\\
    \end{pmatrix}        
$
\\
$   (2) \qquad
	f \bigg (  
    \begin{pmatrix}
    4		\\
    -3		\\
    \end{pmatrix}  
	\bigg )   
	=
	4 f \bigg (  
    \begin{pmatrix}
    1		\\
    0		\\
    \end{pmatrix}  
	\bigg )  
	+
	-3 f \bigg (  
    \begin{pmatrix}
    0		\\
    1		\\
    \end{pmatrix}  
	\bigg ) 
	=
	4 f_x - 3 f_y	
	= 
	\begin{pmatrix}
    -4		\\
    2		\\
    \end{pmatrix}   	
$
Lösning av (1) och (2) ger: \\
$
	f_x	
	= 
	\begin{pmatrix}
    2		\\
    \frac{16}{5}		\\
    \end{pmatrix} 
    ,
    \qquad
    f_y	
	= 
	\begin{pmatrix}
    4		\\
    \frac{18}{5}		\\
    \end{pmatrix}    
    $
\\
\\
Vilket betyder att 
\\
\\
	$
	A = 
	\begin{pmatrix}
    f_x 				&	f_y	
    \end{pmatrix}
    =
	\begin{pmatrix}
    2 				&	4		\\
    \frac{16}{5}	& \frac{18}{5}		\\
    \end{pmatrix} 
    $

\subsection*{6}
\begin{itemize}
    \item[a) ] 
    Arean av parallellogrammet som spänns upp av $\vec{PQ}$ och $\vec{PR}$ ges 
    av: \\
        $\vec{PQ} = \vec{u} = 
        \begin{bmatrix}
            3 & -1 \\
            4 & -2
        \end{bmatrix}$
        $= \begin{bmatrix}
            2 \\
            2
           \end{bmatrix}$
               
        and \\

        $\vec{PR} = \vec{v}= 
        \begin{bmatrix}
         -1  - 1 \\
          6  - 2
        \end{bmatrix}$
        $= \begin{bmatrix}
           -2 \\
            4
           \end{bmatrix}$

        $A = det \begin{bmatrix}
                   \vec{u} & \vec{v}
                 \end{bmatrix}
           = det \begin{bmatrix}
                   2 & -2 \\
                   2 &  4
                 \end{bmatrix}
           = 12$ 

 Aren för vår triangel är $\frac{A}{2} = 6$
    \item[b)] 120 \\
        {\it Hint: $\frac{area(D')}{area(D)} = |det(A)|$}
                        
\end{itemize}

\subsection*{7}
$A = 
\begin{bmatrix}
1  &  0 & 0 \\
0  & -1 & 0 \\
0  &  0 & 1
\end{bmatrix}$
 
$ b = 
\begin{bmatrix}
0 \\
2 \\
0
\end{bmatrix}$

\end{document}
