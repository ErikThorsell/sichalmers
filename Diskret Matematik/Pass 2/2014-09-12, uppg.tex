	%SI 2014-09-05

\documentclass{article}

\usepackage[T1]{fontenc}
\usepackage[utf8]{inputenc}
\usepackage[swedish]{babel}
\usepackage{fullpage}
\usepackage{amssymb}
\usepackage{bussproofs}

\let\emptyset\varnothing

\title{Supplemental Instructions}
\author{Erik Thorsell \\ 
		\small{erithor@student.chalmers.se}
}
\date{2014-09-12}

\begin{document}
\maketitle

\subsection*{Repetition}
Repetition är moder till all inlärning.
\begin{enumerate}
\item[1.] Avgör om följande logiska argument är en tautologi: 
\begin{prooftree}
\Axiom$\neg x \rightarrow \neg w \fCenter\ $
\UnaryInf$(x \lor \neg w) \rightarrow z \fCenter\ $
\UnaryInf$\neg p \rightarrow \neg z \fCenter\ $
\UnaryInf$p \rightarrow (\neg r \lor \neg a) \fCenter\ $
\UnaryInf$\neg r \lor \neg a \fCenter\ $
\end{prooftree}
{\it Uppg 1.7 (h)}

\item[2.] Låt "universum" vara mängden av alla barn på tåget {\it(undertecknad sitter på ett tåg för tillfället)} och låt $P(x) : x$ {\it är irriterande}.\\
Skriv följande utsagor på symbolisk logisk form:
\begin{itemize}
\item[a)] Alla barn på tåget är irriterande.
\item[b)] Några barn på tåget är inte irriterande.
\item[c)] Det finns inte en enda unge på tåget som inte är irriterande.\\
\end{itemize}

\item[3.] Avgör om följande logiska argument är giltigt:\\
Somliga gitarrister gillar blues\\
\underline{Alla groupies gillar blues}\\
Somliga groupies gillar gitarrister
\end{enumerate}

\subsection*{Mängder}
\begin{enumerate}
\item[4.] Skriv elementen i följande mängder:
\begin{itemize}
\item[a)] $\{ x \in \mathbb{Z+} : -3 < x < 3 \}$
\item[b)] $\{ x \in \mathbb{Z} : 3 > x \land x > -1 \} $
\end{itemize}

\item[5.] Låt $A$, $B$ och $C$ vara tre mängder. Givet att $A$ och $B$ är disjunkta,
$|A \cup B \cup C| = 30$, $|A \backslash C|=10$ och $|B \backslash C|=5$. Vad är det högsta, respektive minsta, antal element $C$ kan innehålla?

\item[6.] Låt $A = \{ x \in \mathbb{N} : x < 4\}$
\begin{itemize}
\item[a)] Bestäm potensmängden $\mathcal P \left({A}\right)$
\item[b)] Bestäm den Kartesiska produkten $A \times \mathcal P \left({A}\right)$
\end{itemize}
\end{enumerate}

\subsection*{Funktioner}
\begin{enumerate}
\item[7.] Låt $A$ vara mängden av alla andragradspolynom med reella koefficienter och $B$ mängden av alla förstagradspolynom med reella koefficienter.\\
Derivering är en funktion $D: A \longrightarrow B$ definierad av:
\[D(a+bx+cx^2)=b+2cx\]
\begin{itemize}
\item[a)] Teckna ett uttryck för $A$, resp. $B$.
\item[b)] Är $D: A \longrightarrow B$ injektiv?
\item[c)] Är $D: A \longrightarrow B$ surjektiv?
\item[d)] Har $D: A \longrightarrow B$ invers? {\it Om så är fallet, bestäm inversen.}
\end{itemize}
\end{enumerate}

\subsection*{Relationer}
\begin{enumerate}
\item[8.] Låt $R$ vara relation på $\mathbb{R}^2$ definierad av att $(a,b)R(c,d)$ om $a^2+b^2=c^2+d^2$.
\begin{itemize}
\item[a)] Visa att $R$ är en ekvivalensrelation.
\item[b)] Rita ekvivalensklassen som innehåller (1,1) i ett koordinatsystem.
\item[c)] Beskriv ekvivalensklassen geometriskt.
\item[d)] Ge en mängd med exakt ett element ur varje ekvivalensklass.
\end{itemize}
\end{enumerate}

\subsection*{Operatorer}
\begin{enumerate}
\item[9.] Vi definierar en binär operator $\star$ på $\mathbb{R}$ genom
\[x \star y = x-2y+3xy\]
\begin{itemize}
\item[a)] Visa att $\star$ inte är associativ.
\item[b)] Visa att $\star$ inte är kommutativ.
\item[c)] Vilka par $x,y \in \mathbb{R}$ kommuterar med avseende på $\star$?
\end{itemize}
\end{enumerate}
\end{document}
