%SI 2014-09-05

\documentclass{article}

\usepackage[T1]{fontenc}
\usepackage[utf8]{inputenc}
\usepackage[swedish]{babel}
\usepackage{fullpage}
\usepackage{amssymb}
\usepackage{bussproofs}
\usepackage{amsmath}
\let\emptyset\varnothing

\title{Supplemental Instructions}
\author{Erik Thorsell \\ 
		\small{erithor@student.chalmers.se}
}
\date{2014-10-03}

\begin{document}
\maketitle


\subsection*{Repetition}
Repetition är moder till all inlärning.
\begin{enumerate}

\item[1.]
Vilka av följande par av logiska formler är ekvivalenta? Bevisa eller ge 
motexempel:
\begin{itemize}
    \item[a) ] $\neg (p \land q)$ och $\neg p \lor \neg q$
    \item[b) ] $p \rightarrow q$ och $q \rightarrow p$
    \item[c) ] $(p \rightarrow q) \land (q \rightarrow r)$ och $p \rightarrow r$
\end{itemize}
{\it Tentamen 2002-08-21}

\item[2.]
Funktionen $f:[0,1] \rightarrow B$ given av $$f(x) = \sqrt{1-x^2}$$ är surjektiv.
\begin{itemize}
    \item[a) ] Ange $B$.
    \item[b) ] Visa att $f$ även är injektiv och beräkna dess invers.
\end{itemize}
{\it Tentamen 2003-10-23}
\end{enumerate}


\subsection*{Diofantiska Ekvationer}
\begin{enumerate}

\item[3.]
Emil och Emilia köper marsipangrisar och chokladaskar för 29 respektive 37 
kronor styck. Emil spenderar 533 kronor på detta medan Emilia gör av med 491 
kronor.\\
Vem köpte flest chokladaskar?\\

{\it Tentamen 2004-01-16}

\item[4.]
Lös den diofantiska ekvationen $$45x+50y=25$$ fullständigt.\\

{\it Tentamen 2005-08-16}
\end{enumerate}


\subsection*{Kongruens}
\begin{enumerate}

\item[5.]
Vad blir $8^{20}+13^{41}$ i $\mathbb{Z}_{25}$?\\

{\it Tentamen 2005-08-16}

\item[6.]
Ungefär 1 500 enkronor staplas på ett bord. När enkronorna läggs i staplar med 
10 mynt i varje blir det 7 mynt över, när staplarna innehåller 7 mynt var blir 
det 3 mynt över och när staplarna består av 13 mynt blir det 10 mynt över.\\
Hur många enkronor finns det på bordet?\\

{\it Tentamen 2002-01-17}

\end{enumerate}
\end{document}
