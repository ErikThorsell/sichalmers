%SI 2014-09-05

\documentclass{article}

\usepackage[T1]{fontenc}
\usepackage[utf8]{inputenc}
\usepackage[swedish]{babel}
\usepackage{fullpage}
\usepackage{amssymb}
\usepackage{bussproofs}
\usepackage{amsmath}
\usepackage{graphicx}
\usepackage{verbatim}
\usepackage{tikz}
\let\emptyset\varnothing


\title{Supplemental Instructions}
\author{Benjamin Eriksson \& Erik Thorsell \\ 
		\small{beneri@student.chalmers.se} \&
		\small{erithor@student.chalmers.se}
}
\date{
      %Place date here!
     }

\begin{document}
\maketitle


\section*{Repetition}
Låt f vara en linjär avbildning i planet som uppfyller:
\[
	f \bigg (  
    \begin{pmatrix}
    2		\\
    1		\\
    \end{pmatrix}  
	\bigg )   
	= 
	\begin{pmatrix}
    8		\\
    10		\\
    \end{pmatrix}    
    \qquad 
	och
	\qquad
    f \bigg (  
    \begin{pmatrix}
    4		\\
    -3		\\
    \end{pmatrix}  
	\bigg )   
	= 
	\begin{pmatrix}
    -4		\\
    2		\\
    \end{pmatrix}   	
\]
Bestäm matrisen A som är sådan att $f = f_A$, dvs matrisavbildningen map A.

\section*{Area och Volymförändringar}
\subsection*{1.}
Låt $P=(1,2)$, $Q=(3,4)$, $R=(-1,6)$.
\begin{itemize}
    \item[a) ] Vad är arean av triangeln $\bigtriangleup PQR$?
    \item[b) ] Låt $f$ vara den linjära avbildningen med matrisen: 
               $$\begin{bmatrix}
               2  & 3 \\
               8  & 2
               \end{bmatrix}$$
               Vad är arean av bilden $f(\bigtriangleup PQR)$, av triangeln 
               $\bigtriangleup PQR$?
\end{itemize}

\section*{Affina avbildningar}
\subsection*{2.}
Bestäm en matris $A$ och en vektor ${\bf b}$ så att den affina avbildning $f$ 
som är spegling av punkterna i rummet i planet som ges av $y=1$ ges av: 
$$f({\bf x})=A{\bf x}+{\bf b}$$

\section*{Geometrin hos linjära avbildningar}
\subsection*{3.}
\begin{itemize}
\item[a) ] Bestäm matrisen för den linjära avbildningen i planet som ges av spegling i linjen $y = kx$ där 
$k \in R$
\item[b) ] Var skulle vektorn $v = (3, 7)$ avbildas om $y = 2x$? 
\end{itemize}

\section*{Sammansatta avbildningar}
\subsection*{4.}
\begin{itemize}
\item[a) ] Bestäm matrisen $A$ för den linjära avbildningen i planet som består av en rotation av pi/6 radianer moturs kring origo.
\item[b) ] Bestäm matrisen $B$ för den linjära avbildningen i planet som ges av spegling i linjen y = x.
\item[c) ] Bestäm också matrisen för rotationen i a) följt av speglingen i b).
\end{itemize}


\section*{Matriser i $n$ dimensioner}
\subsection*{5}
Visa att matrisen för spegling i linje $L$ med riktningsvektor $\mathbf{v}$ i 
$\mathbb{R}^n$ ges av $$A = \frac{2}{\parallel \mathbf{v}^2 \parallel}
(\mathbf{v} \mathbf{v}^t)-I$$

\section*{Linjära Ekvationssytem}
\subsection*{6}
Skriv ekvationssystemet 
\begin{equation}
    \begin{cases}
                    x+2y-z=3 \\
                    2x+2y-z=4 \\
                    2x+5y+2z=2
    \end{cases}
\end{equation} 
\noindent
\begin{itemize}
    \item[a) ] som en matrisekvation $A \mathbf{x} = \mathbf{b}$.
    \item[b) ] Lös ekvationssystemet mha Gausselimination.
\end{itemize}

\end{document}
