%SI 2014-09-05

\documentclass{article}

\usepackage[T1]{fontenc}
\usepackage[utf8]{inputenc}
\usepackage[swedish]{babel}
\usepackage{fullpage}
\usepackage{amssymb}
\usepackage{bussproofs}
\usepackage{amsmath}
\usepackage{graphicx}
\usepackage{verbatim}
\usepackage{tikz}
\let\emptyset\varnothing


\title{Supplemental Instructions}
\author{Benjamin Eriksson \& Erik Thorsell \\ 
		\small{beneri@student.chalmers.se} \&
		\small{erithor@student.chalmers.se}
}
\date{
      %Place date here!
     }

\begin{document}
\maketitle


\section*{Repetition}
Ett plan går genom punkterna $A=(1,1,-2)$, $B=(-1,5,2)$ och $C=(3,0,2)$.
Bestäm planets ekvation.\\

\section*{Matriser}
\subsection*{1.}
\begin{itemize}
    \item[a) ]	
    			$ 
    			\begin{bmatrix}
    			1 & 2 \\
    			5 & 9
    			\end{bmatrix}
    			+
    			\begin{bmatrix}
    			-2 & 4 \\
    			6  & 1
    			\end{bmatrix}
    			$
    \item[b) ]	
    			$ 
    			\begin{bmatrix}
    			5 & -7 \\
    			4 & -1
    			\end{bmatrix}
    			-
    			\begin{bmatrix}
    			5   & 5 \\
    			-2  & 8
    			\end{bmatrix}
    			$
    			
    \item[c) ]	
    			$ 
    			\begin{bmatrix}
    			5  & 5 \\
    			2  & 3
    			\end{bmatrix}
    			\cdot
    			\begin{bmatrix}
    			2   \\
    			7  
    			\end{bmatrix}
    			$
    			
    \item[d) ]	
    			$ 
    			\begin{bmatrix}
    			5  & 5 \\
    			2  & 3
    			\end{bmatrix}
    			\cdot
    			\begin{bmatrix}
    			2 & 8  \\
    			7 & 1
    			\end{bmatrix}
    			$
    \item[e) ]	
    			$ 
    			\begin{bmatrix}
    			5 & 7  & 3 \\
    			4 & -6 & 9
    			\end{bmatrix}
    			^{T}
    			$
    		
\end{itemize}

\subsection*{2.}
\begin{itemize}
    \item[a) ]	Beräkna determinanten.
				\newline
				\\
				$ 
    			\begin{vmatrix}
   		 		7  & 4 \\
  		  		1  & 2
		    	\end{vmatrix}
  			  	$
    \item[b) ]	Vad kan sägas om vinkeln mellan vektorerna 
    			$
    			u =   			  
				\begin{pmatrix}
   		 		7  \\
  		  		1  
		    	\end{pmatrix}  		
		    	,
		    	v =   			  
				\begin{pmatrix}
   		 		4  \\
  		  		2  
		    	\end{pmatrix}  			
		    	$ 
		    	utifrån determinanten?
\end{itemize}

\subsection*{3.}
\begin{itemize}
    \item[a) ]	
				$ 
    			\begin{bmatrix}
    			7  & 2 \\
    			3  & 5
    			\end{bmatrix}
    			\cdot
    			\begin{bmatrix}
    			1 & 0  \\
    			0 & 1
    			\end{bmatrix}
    			$
    			
    \item[b) ]	Beräkna inversen 
    			\newline
    			\\
    			$ 
    			\begin{bmatrix}
    			7  & 2 \\
    			3  & 5
    			\end{bmatrix}
				^{-1}
    			$
    			
    \item[c) ]	
    			$ 
    			\begin{bmatrix}
    			7  & 2 \\
    			3  & 5
    			\end{bmatrix}
    			\cdot
    			\begin{bmatrix}
    			7  & 2 \\
    			3  & 5
    			\end{bmatrix}
				^{-1}
    			$
    			
    \item[d) ]	Bevisa att
    			$ 
    			A = 
    			\begin{bmatrix}
    			a  & b \\
    			c  & d
    			\end{bmatrix}  
    			\implies
    			A^{-1} = 
    			\frac{1}{det(A)}
    			\begin{bmatrix}
    			d  & -b \\
    			-c & a
    			\end{bmatrix}		
    			$
    			\\
    			\\
    			\it{Hint: $A A^{-1} = ... $}
\end{itemize}

\section*{Linjära avbildningar}
\subsection*{4.}
Låt 
\[
	D = 
    \begin{bmatrix}
    5	&	2	&	-1	\\
    6	&	3	&	7	\\
    -3	&	2	&	3
    \end{bmatrix}  
    ,\>
    \vec{u} =
    \begin{bmatrix}
    2   	\\
   	-1 		\\
   	3		
   	\end{bmatrix}
   	,\>
    \vec{v} =
    \begin{bmatrix}
    4   	\\
   	3 		\\
   	-1		
   	\end{bmatrix}
\]
och låt $f_D$ vara matrisavbildningen m a p D. Beräkna
\begin{itemize}
	\item[a) ]	$f_{D}(\vec{u})$
	\item[b) ]	$f_{D}(\vec{v})$
	\item[c) ]	$f_{D}(\vec{u} + \vec{v})$
	\item[d) ]	$f_{D}(\vec{2u})$
\end{itemize}

\subsection*{5.}
Låt f vara en linjär avbildning i planet som uppfyller:
\[
	f \bigg (  
    \begin{pmatrix}
    2		\\
    1		\\
    \end{pmatrix}  
	\bigg )   
	= 
	\begin{pmatrix}
    8		\\
    10		\\
    \end{pmatrix}    
    \qquad 
	och
	\qquad
    f \bigg (  
    \begin{pmatrix}
    4		\\
    -3		\\
    \end{pmatrix}  
	\bigg )   
	= 
	\begin{pmatrix}
    -4		\\
    2		\\
    \end{pmatrix}   	
\]
Bestäm matrisen A som är sådan att $f = f_A$, dvs matrisavbildningen map A.



\section*{Area och Volymförändringar}
\subsection*{6.}
Låt $P=(1,2)$, $Q=(3,4)$, $R=(-1,6)$.
\begin{itemize}
    \item[a) ] Vad är arean av triangeln $\bigtriangleup PQR$?
    \item[b) ] Låt $f$ vara den linjära avbildningen med matrisen: 
               $$\begin{bmatrix}
               2  & 3 \\
               8  & 2
               \end{bmatrix}$$
               Vad är arean av bilden $f(\bigtriangleup PQR)$, av triangeln 
               $\bigtriangleup PQR$?
\end{itemize}

\section*{Affina avbildningar}
\subsection*{7.}
Bestäm en matris $A$ och en vektor ${\bf b}$ så att den affina avbildning $f$ 
som är spegling av punkterna i rummet i planet som ges av $y=1$ ges av: 
$$f({\bf x})=A{\bf x}+{\bf b}$$


\section*{Lösningar}
\section*{1}
\subsection*{a)}
Addera cellvis.
$ 
    			\begin{bmatrix}
    			1-2 & 2+4 \\
    			5+6 & 9+1
    			\end{bmatrix}
    			=
    			\begin{bmatrix}
    			-1 & 6 \\
    			11  & 10
    			\end{bmatrix}
    			$
    			
\subsection*{b)}
Subtrahera cellvis.
$ 
    			\begin{bmatrix}
    			5-5 & -7-5 \\
    			4-(-2) & -1-8
    			\end{bmatrix}
    			=
    			\begin{bmatrix}
    			0 & -12 \\
    			6  & -9
    			\end{bmatrix}
    			$
    			
\subsection*{c)}
Tänk rad gånger kolumn.
$ 
    			\begin{bmatrix}
    			2*5+5*7  \\
    			2*2+3*7
    			\end{bmatrix}
    			=
    			\begin{bmatrix}
    			45  \\
    			25
    			\end{bmatrix}
    			$

\subsection*{d)}

$ 
    			\begin{bmatrix}
    			2*5+5*7 & 5*8+5*1  \\
    			2*2+3*7 & 2*8+3*1
    			\end{bmatrix}
    			=
    			\begin{bmatrix}
    			45 & 45\\
    			25 & 19
    			\end{bmatrix}
    			$

\subsection*{e)}
Här byter vi plats på raderna och kolumnerna.
				$ 
    			\begin{bmatrix}
    			5 & 4 \\
    			7 & -6 \\
    			9 & 9 
    			\end{bmatrix}
    			$
    			
    			
\section*{2}
\subsection*{a)}
				$ 
    			\begin{vmatrix}
   		 		7  & 4 \\
  		  		1  & 2
		    	\end{vmatrix}
		    	=
		    	7*2 - 1*4
		    	= 10
  			  	$
\subsection*{a)}
$Det(A) = 0 \implies \vec{u}$ och $\vec{v}$ är linjärt oberoende. Vilket betyder att vinkeln är skild från 0 och 180.

\section*{3}
\subsection*{a) }
	$ 
    			\begin{bmatrix}
    			7  & 2 \\
    			3  & 5
    			\end{bmatrix}
    			\cdot
    			\begin{bmatrix}
    			1 & 0  \\
    			0 & 1
    			\end{bmatrix}
    			=
    			\begin{bmatrix}
    			7  & 2 \\
    			3  & 5
    			\end{bmatrix}
    			$
    			
\subsection*{b) }
				$
				\begin{bmatrix}
    			7  & 2 \\
    			3  & 5
    			\end{bmatrix}
				^{-1}
				=
				\frac{1}{7*5-2*3}
				\begin{bmatrix}
    			5  & -2 \\
    			-3  & 7
    			\end{bmatrix}
				$
\subsection*{c) }
$				A A^{-1} = I =
				\begin{bmatrix}
    			1  & 0 \\
    			0  & 1
    			\end{bmatrix}
    			$ 
\subsection*{d) }
				$
				\begin{bmatrix}
    			a  & b \\
    			c  & d
    			\end{bmatrix}  
    			\cdot
    			\frac{1}{det(A)}
    			\begin{bmatrix}
    			d  & -b \\
    			-c & a
    			\end{bmatrix}
    			=
    			\frac{1}{det(A)}
    			\begin{bmatrix}
    			ad - bc  & -ab + ab \\
    			cd - cd  & -bc + ad
    			\end{bmatrix} 
    			=
    			\frac{1}{ad-bc}
    			\begin{bmatrix}
    			ad - bc  & -ab + ab \\
    			cd - cd  & -bc + ad
    			\end{bmatrix} 
    			=
    			\begin{bmatrix}
    			1 & 0 \\
    			0 & 1
    			\end{bmatrix} 
    			$

\section*{4}
\begin{itemize}
	\item[a) ]	$f_{D}(\vec{u}) 
				=
				\begin{bmatrix}
    			5	&	2	&	-1	\\
    			6	&	3	&	7	\\
    			-3	&	2	&	3
    			\end{bmatrix}   
    			\cdot
    			\begin{bmatrix}
    			2   	\\
   				-1 		\\
   				3		
   				\end{bmatrix}
   				=
				\begin{bmatrix}
    			5 \\
    			30 \\
    			1
    			\end{bmatrix} 				
				$
				
	\item[b) ]	$f_{D}(\vec{v}) 
				=
				\begin{bmatrix}
    			5	&	2	&	-1	\\
    			6	&	3	&	7	\\
    			-3	&	2	&	3
    			\end{bmatrix}   
    			\cdot
    			\begin{bmatrix}
    			2   	\\
   				3 		\\
   				-1		
   				\end{bmatrix}
   				=
				\begin{bmatrix}
    			27   	\\
   				26		\\
   				-9		
   				\end{bmatrix} 				
				$
	\item[c) ]	$f_{D}(\vec{u} + \vec{v})
				=
				f_{D}(\vec{u}) + f_{D}(\vec{v})	=
				\begin{bmatrix}
    			32   	\\
   				56		\\
   				-8		
   				\end{bmatrix} 	
   				$
	\item[d) ]	$f_{D}(\vec{2u})
				=
				2f_{D}(\vec{u})
				=
				\begin{bmatrix}
    			10 \\
    			60 \\
    			2
    			\end{bmatrix}
				$
\end{itemize}

\section*{5}
Dela upp i enhetsvektorer.
\\
$
	(1) \qquad
	f \bigg (  
    \begin{pmatrix}
    2		\\
    1		\\
    \end{pmatrix}  
	\bigg )   
	=
	2 f \bigg (  
    \begin{pmatrix}
    1		\\
    0		\\
    \end{pmatrix}  
	\bigg )  
	+
	f \bigg (  
    \begin{pmatrix}
    0		\\
    1		\\
    \end{pmatrix}  
	\bigg ) 
	=
	2 f_x + f_y	
	=
	\begin{pmatrix}
    8		\\
    10		\\
    \end{pmatrix}        
$
\\
$   (2) \qquad
	f \bigg (  
    \begin{pmatrix}
    4		\\
    -3		\\
    \end{pmatrix}  
	\bigg )   
	=
	4 f \bigg (  
    \begin{pmatrix}
    1		\\
    0		\\
    \end{pmatrix}  
	\bigg )  
	+
	-3 f \bigg (  
    \begin{pmatrix}
    0		\\
    1		\\
    \end{pmatrix}  
	\bigg ) 
	=
	4 f_x - 3 f_y	
	= 
	\begin{pmatrix}
    -4		\\
    2		\\
    \end{pmatrix}   	
$
Lösning av (1) och (2) ger: \\
$
	f_x	
	= 
	\begin{pmatrix}
    2		\\
    \frac{16}{5}		\\
    \end{pmatrix} 
    ,
    \qquad
    f_y	
	= 
	\begin{pmatrix}
    4		\\
    \frac{18}{5}		\\
    \end{pmatrix}    
    $
\\
\\
Vilktet betyder att 
\\
\\
	$
	A = 
	\begin{pmatrix}
    f_x 				&	f_y	
    \end{pmatrix}
    =
	\begin{pmatrix}
    2 				&	4		\\
    \frac{16}{5}	& \frac{18}{5}		\\
    \end{pmatrix} 
    $
\end{document}
