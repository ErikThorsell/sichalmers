%SI 2014-09-05

\documentclass{article}

\usepackage[T1]{fontenc}
\usepackage[utf8]{inputenc}
\usepackage[swedish]{babel}
\usepackage{fullpage}
\usepackage{amssymb}
\usepackage{bussproofs}
\usepackage{amsmath}
\let\emptyset\varnothing

\title{Supplemental Instructions}
\author{Erik Thorsell \\ 
		\small{erithor@student.chalmers.se}
}
\date{2015-10-8}

\begin{document}
\maketitle


\subsection*{Repetition}
Repetition är moder till all inlärning.
\begin{enumerate}

\item[1.]
Ge fullständig lösning till den diofantiska ekvationen $$28x+36y=100$$ förkorta 
så långt som möjligt. Ange även inversen till 7 i $\mathbb{Z}_9$.
\\ \\ {\it Tentamen 2005-01-11}\\

\item[2.]
Emilia har en stor släkt, så för att göra julklappsinköpen enklare ger hon alla 
varsin marsipangris. Om hon köper marsipangrisar av en enkel sort för 29 kronor 
styck kan alla de närmaste släktingarna - plus kusinerna på mammas sida - få 
varsin och Emilia får 3 kronor över. Om hon istället satsar på en finare sorts 
grisar för 40 kronor styck räcker pengarna till att de närmaste släktingarna - plus 
kusinerna på pappas sida - ska få varsin. Det blir då 8 kronor över.\\ 
Emilia har mindre än 1000 kronor. Hur mycket pengar har hon? Hur många fler är 
kusinerna på mammas sida än kusinerna på pappas sida?
\\ \\{\it Tentamen 2004-12-17}\\

\item[3.]
Visa att det för alla positiva heltal $n$ gäller att 
$$\sum_{k=0}^{n-1} (3k^2+3k+1)=n^3$$
{\it Tentamen 2005-08-16}\\
\end{enumerate}

\subsection*{Kombinatorik}
\begin{enumerate}

\item[4.]
Hur många ``ord'' kan man bilda av bokstäverna i
\begin{itemize}
    \item[a) ] KAFFEKOPP
    \item[b) ] LINUX
    \item[c) ] LUFTMADRASS
\end{itemize}
{\it Tentamen 2005-03-30}

\item[5.]
Hur många sjusiffriga tal där alla siffror är 1, 2 eller 3 finns det:
\begin{itemize}
    \item[a) ] Totalt?
    \item[b) ] Som innehåller exakt två ettor?
    \item[c) ] Som saknar ettor?
    \item[a) ] Som innehåller minst fyra ettor?
\end{itemize}
{\it Tentamen 2004-10-22}\\

\item[6.]
Emil, Emilia och Gösta ska dela elva karameller mellan sig. 
\begin{itemize}
    \item[a) ] På hur många sätt kan de fördela karamellerna?
    \item[b) ] Givet att ingen av dem helt ska bli utan karamell?
\end{itemize}
{\it Tentamen 2005-08-16}

\item[7.]
Poker spelas med en vanlig kortlek med 52 kort: Tretton valörer i fyra olika färger.
En pokerhand har fem kort, hur många pokerhänder finns det:
\begin{itemize}
    \item[a) ] Totalt?
    \item[b) ] Som innehåller ett fyrtal ({\it dvs. fyra kort i samma valör})?
    \item[c) ] Som innehåller en kåk ({\it dvs. tre kort i en valör och två 
               kort i en annan valör})?
    \item[d) ] Som innehåller ett tvåpar ({\it dvs. två kort i en valör, 
               två kort i en annan valör, och ett kort i en tredje valör})?
\end{itemize}
{\it Tentamen 2005-10-21}
\end{enumerate}

\subsection*{RSA}
\begin{enumerate}

\item[8.]
Antag att du är användare i ett RSA-krypterat system och att din privata 
nyckel består av de två primtalen $p=5$ och $q=11$ samt talet $s=23$ 
(vilket är relativt primt $\Phi (pq)$). Således består din din publika nyckel 
av talen $55$ och $23$.\\
Antag nu att någon skickar dig ett meddelande som är krypterat för att ingen 
annan än du ska kunna läsa det. Om meddelandet är 3, vad är då det riktiga 
meddelandet?
\\ \\{\it Tentamen 2004-10-22}

\end{enumerate}
\end{document}
