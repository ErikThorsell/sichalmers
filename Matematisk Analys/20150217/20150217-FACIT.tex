%SI 2014-09-05

\documentclass{article}

\usepackage[T1]{fontenc}
\usepackage[utf8]{inputenc}
\usepackage[swedish]{babel}
\usepackage{fullpage}
\usepackage{amssymb}
\usepackage{bussproofs}
\usepackage{amsmath}
\usepackage{graphicx}
\usepackage{verbatim}
\usepackage{tikz}
\let\emptyset\varnothing


\title{Supplemental Instructions}
\date{
      %Place date here!
     }

\begin{document}
\maketitle

\section*{1}
\begin{itemize}
    \item[a) ] 	$u  = x^2$ \\ 
    			$\frac{du}{dx} = 2x$ \\
    			$\frac{du}{2x} = dx$ \\
    			$\int{ xe^{x^2} dx } = \int{\frac{ xe^{u} du}{2x} } = 
    			\int{\frac{ e^{u} du}{2} } = 1/2 e^u = 1/2 e^{x^2}$ \\
    			
    \item[b) ]  Här gör vi en lite annorlunda substitution: \\
    			$x = sin(u) \> \implies \> 
    			\frac{dx}{du} = \frac{d}{du} sin(u) = cos(u)$ \\
    			$dx = cos(u) du$ \\

    		  	$\int \frac{1}{\sqrt{1-(sin u)^2}} cos(u) du$ \\ 
    			$\int \frac{1}{cos(u)} cos(u) du$ \\ 
    			$\int 1 du = u$ \\ 
    			$x = sin(u) \implies u = sin^{-1}(x) $ \\ \\
    			$\int \frac{1}{\sqrt{1-x^2}} dx = sin^{-1}(x)$ 
    			
\end{itemize}

\section*{2}
\begin{itemize}
    \item[a) ]	Man bör alltid skriva om integralen men limits för att göra den 				giltlig: \\ 
				$\lim_{b \rightarrow \inf} \> \int_{1}^{b} \frac{1}{x^2} dx$ \\
    			$\int_{1}^{b} \frac{1}{x^2} dx = (- 1/b) - (- 1/1)
    			= \{  \lim_{b \rightarrow \inf} \} = (- 0) - (-1) = 1$ \\
    			
    			
    \item[b) ] 	Från formelsamlingen ser vi att: \\
    			$\int \frac{1}{1+x^2} dx = tan^{-1}(x)$ \\
    			Alltså får vi: \\
    			$tan^{-1}(\infty) - tan^{-1}(0) = \frac{\pi}{2} - 0 = \frac{\pi}{2}$
\end{itemize}

\section*{3}
\begin{itemize}
    \item[a) ] $u = x^2+1 \\
                du = 2xdx \\
                \frac{1}{2} \int cos(u) du = \\
                \frac{sin(u)}{2} + c = \\
                \frac{1}{2} sin(x^2+1) + c$ 
    \item[b) ] $u = 6x^3 + 5 \\
                du = 18x^2dx \\
                \int 18x^2 \sqrt[4]{6x^3+5}dx = \\ 
                \int (6x^3+5)^{\frac{1}{4}}(18x^2dx) = \\
                \int u^{\frac{1}{4}}du = \\
                \frac{4}{5} u^{\frac{5}{4}}+c = \\
                \frac{4}{5}(6x^3+5)^{\frac{5}{4}}+c$ 
    \item[c) ] $\frac{2^{x^3+1}}{ln 8} + C$ 
\end{itemize}



\section*{4}
Om vi kikar på integralen som ges på s. 368: $\int_a^b f(x) dx$, så är: \\
a = 0, b = $\pi$, $f(x) = sin(x)$, $x_i = a+ih$\\\\

\noindent
a) \\
\noindent
$n = 10 \Rightarrow h = \frac{b-a}{n} = \frac{\pi}{10}$ och 

\begin{tabular}{l|l*{6}{c}} \\
$i$     & 0 & 1 & 2 & 3 & ...  & 9 & 10 \\
\hline
$x_i$   & 0 & $\frac{\pi}{10}$ & $\frac{2 \pi}{10}$ & $\frac{3 \pi}{10}$ & 
        ... & $\frac{9 \pi}{10}$ & $\frac{10 \pi}{10}$  \\
        [5pt]
$f_i$   & $sin(0)$ & $sin(\frac{\pi}{10})$ & $sin(\frac{2 \pi}{10})$ & 
        $sin(\frac{3 \pi}{10})$ & ... & $sin(\frac{9 \pi}{10})$ & 
        $sin(\frac{10 \pi}{10})$ \\
\end{tabular}\\\\
[20pt]
$\int_0^{\pi} sin(x) dx \approx \frac{h}{2}[f_0 + 2f_1 + 2f_2 + 2f_3 + ... + 2f_9 + f_{10}]
 = 1.983 523 538$\\\\
Error = $0.824\%$\\\\
\noindent
b) \\
\noindent
$n = 20 \Rightarrow h = \frac{\pi}{20}$\\\\
$\int_0^{\pi} sin(x) dx \approx \frac{h}{2}[f_0 + 2f_1 + 2f_2 + 2f_3 + ... + 2f_{19} + f_{20}]
= 1.995 885 973$\\\\
Error = $0.206\%$


\section*{5}
$\frac{\pi}{5}$ cu. units

\clearpage
\section*{6}
\begin{itemize}
	\item[a) ] $y = \frac{4}{3} \> x$, där x går från 0 till 3. \\
			 Här får vi en vanlig triangel med basen 3 och höjden 4. Längden blir således 5, enl pythagoras. 
			 
	\item[b) ] 	$y = \frac{2}{3} (x-1)^{ \frac{3}{2} }$, där x går från 0 till 4 \\
				Här måste vi använda längdformeln. 
				$ds = \sqrt{1 + f'(x)^2}$ \\
				$ds = \sqrt{1 + x-1} $ \\
				$ds = \sqrt{x} $ \\
				$L = \int_{0}^{4} ds = \int_{0}^{4} \sqrt{x} dx $ \\
				$L =  2/3 ( 4^{3/2} - 0^{3/2} ) = 2/3 * 8 = 16/3 $
\end{itemize}
\end{document}
