%SI 2014-09-05

\documentclass{article}

\usepackage[T1]{fontenc}
\usepackage[utf8]{inputenc}
\usepackage[swedish]{babel}
\usepackage{amssymb}
\usepackage{fullpage}
\usepackage{bussproofs}
\usepackage{amsmath}
\usepackage{graphicx}
\usepackage{verbatim}
\usepackage{tikz}

\usepackage{tkz-berge}
\let\emptyset\varnothing



\title{Supplemental Instructions}
\author{Benjamin Eriksson \& Erik Thorsell \\ 
		\small{beneri@student.chalmers.se} \&
		\small{erithor@student.chalmers.se}
}

\date{
     2014-12-16 
 }

\begin{document}
\maketitle

\subsection*{1}
Beräkna 

$$
det
\begin{bmatrix}
 1 & 2 & -1 \\
 2 & 5 & 3 \\
 3 & 7 & 1
\end{bmatrix}
$$

\subsection*{2}
Beräkna $A^{-1}$, där 
$
A = 
\begin{bmatrix}
1 & 3 & -1 \\
0 & 1 & 2 \\
1 & 0 & -1
\end{bmatrix}
$

\subsection*{3}
Beräkna $A^n$ för matrisen
$$
A = 
\begin{bmatrix}
 2 & 3 \\
 0 & 5 
\end{bmatrix}
$$

\subsection*{4}
Beräkna matrisen för den linjära avbildning i rummet som består av rotation 
$\frac{\pi}{4}$ radianer runt $x$-axeln, i den riktning som bestäms av att 
positiva delen av $y$-axeln vrids mot den positiva delen av $z$-axeln, följt 
av projektion på $xz$-planet.

\subsection*{5}
Betrakta ON-basen $F = ({\bf f}_1 {\bf f}_2 {\bf f}_3)$, där
$$
    {\bf f}_1 = \frac{1}{\sqrt{3}} 
        \begin{pmatrix}
            1 \\
           -1 \\
            1 \\
        \end{pmatrix}
        ,
    {\bf f}_2 = \frac{1}{\sqrt{2}} 
        \begin{pmatrix}
            1 \\
            0 \\
           -1 \\
        \end{pmatrix}
        ,
    {\bf f}_3 = \frac{1}{\sqrt{6}} 
        \begin{pmatrix}
            1 \\
            2 \\
            1 \\
        \end{pmatrix}
$$
\noindent
Låt $g$  vara den linjära avbildning som består av spegling i planet genom 
origo som spänns upp av ${\bf f}_1$ och ${\bf f}_2$.
\begin{itemize}
    \item[a) ] Bestäm matrisen för $g$ med avseende på basen $F$.
    \item[b) ] Bestäm matrisen för $g$ med avseende på standardbasen.
\end{itemize}

\subsection*{6}
Bestäm (minsta) avståndet från punkten $P=(2,3,-1)$ till planet $x - 2y + 3z = 7$.

\end{document}
