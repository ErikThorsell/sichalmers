%SI 2014-09-05

\documentclass{article}

\usepackage[T1]{fontenc}
\usepackage[utf8]{inputenc}
\usepackage[swedish]{babel}
\usepackage{fullpage}
\usepackage{amssymb}
\usepackage{bussproofs}
\usepackage{amsmath}
\usepackage{graphicx}
\usepackage{verbatim}
\usepackage{tikz}
\let\emptyset\varnothing


\title{Supplemental Instructions}
\date{
      %Place date here!
     }

\begin{document}
\maketitle

\section*{Repetition}
\begin{itemize}
\item[a) ] Skalärprodukten är störst då vektorerna pekar i samma riktning.
\item[b) ] Vektorprodukten är störst då vektorerna är ortogonala.
\item[c) ] Kvadratisk och nollskild determinant.
\item[d) ] Determinanten är lika med 0.
\item[e) ] $\mathbf{u} \cdot (\mathbf{v} \times \mathbf{w})$
\item[f) ] $f_{A}(c\vec{u} + d\vec{v}) = c f_{A}(\vec{u}) + d f_{A}(\vec{v})$

\end{itemize}


\section*{1}
Genom att beräkna determinanten av matrisen med de tre vektorerna som rader 
ser vi att denna är nollskiljd. Detta är fallet {\bf om och endast om} 
vektorerna är linjärt oberoende.


\section*{2}
$$
\begin{bmatrix}
    1  &  2  &  1  & -1  &  2 \\
    3  &  4  &  5  &  2  &  0 \\
    2  &  2  &  1  &  0  &  2 \\
\end{bmatrix} \rightarrow 
\begin{bmatrix}
    1  &  2  &  1  & -1  &  2 \\
    0  & -2  &  2  &  5  & -6 \\
    0  & -2  & -1  &  2  & -2 \\
\end{bmatrix} \rightarrow
\begin{bmatrix}
    1  &  2  &  1  & -1  &  2 \\
    0  & -2  &  2  &  5  & -6 \\
    0  &  0  & -3  & -3  &  4 \\
\end{bmatrix} \rightarrow
$$

$$
\begin{bmatrix}
    1  &  2  &  1  & -1  &  2 \\
    0  &  1  & -1  &  \frac{-5}{2}  & 3 \\
    0  &  0  &  1  &  1  &  \frac{-4}{3} \\
\end{bmatrix} \rightarrow
\begin{bmatrix}
    1  &  0  &  3  & 4  &  -4 \\
    0  &  1  & -1  &  \frac{-5}{2}  & 3 \\
    0  &  0  &  1  &  1  &  \frac{-4}{3} \\
\end{bmatrix} \rightarrow
\begin{bmatrix}
    1  &  0  &  0  &  1  &  0 \\
    0  &  1  &  0  &  \frac{-3}{2}  & \frac{5}{3} \\
    0  &  0  &  1  &  1  &  \frac{-4}{3} \\
\end{bmatrix}
$$

\noindent
Från detta får vi en lösning med två fria parametrar och alltså geometrisk ett 
plan i $\mathbb{R}^5$. En parametrisering av planet ges av:
$$
\mathbf{x} = s
\begin{bmatrix}
    -1 \\
    3/2 \\
    -1 \\
    1 \\
    0 \\
\end{bmatrix}
+ t
\begin{bmatrix}
    0 \\
    -5/3 \\ 
    4/3 \\
    0 \\
    1 \\
\end{bmatrix}
$$
\section*{3}
Vi kan skriva avbildningen som $f(x) = t_u \circ S \circ t_{-u} = t_u (S ( t_{-u} ) )$, där S är den linjära speglingsmatrisen för $y = 3x$ och $u$ vår förskjutning från origo. I det här fallet är allt flyttat
upp två steg och därför är 
$
u =
\begin{pmatrix}
	0 \\
	2 
\end{pmatrix}
$   \\
S vet vi är 
$
\frac{1}{10}
\begin{pmatrix}
	1	&	3 \\
	3	&	9 \\
\end{pmatrix}
$. 
Så nu stoppar vi in allt i formeln.
$$
f(x) =  t_u (S ( t_{-u} ) ) =  t_u (S ( x - u ) ) = t_u ( Sx - Su  ) = Sx - Su + u
$$
Här ser vi att $A = S$ och $\mathbf{b}  = -Su + u$. 
$$
\mathbf{b}  = 
-Su + u = 
\frac{1}{10}
\begin{pmatrix}
	1	&	3 \\
	3	&	9 \\
\end{pmatrix}
\cdot
\begin{pmatrix}
	0 \\
	2 \\
\end{pmatrix}
+
\begin{pmatrix}
	0 \\
	2 \\
\end{pmatrix}
=
\frac{1}{5}
\begin{pmatrix}
	-3 \\
	1 \\
\end{pmatrix}
$$
\\
Svar: 
$A = 
\frac{1}{10}
\begin{pmatrix}
	1	&	3 \\
	3	&	9 \\
\end{pmatrix}
$
, 
\qquad
$\mathbf{b} = 
\frac{1}{5}
\begin{pmatrix}
	-3 \\
	1 \\
\end{pmatrix}
$


\section*{4}
$$
\begin{bmatrix}
    1 & -2 & 0 & 1 & 0 & 0\\
    0 &  1 & 1 & 0 & 1 & 0\\
    2 &  0 & 1 & 0 & 0 & 1\\
\end{bmatrix}
\rightarrow
\begin{bmatrix}
    1 & -2 & 0 & 1 & 0 & 0\\
    0 &  1 & 1 & 0 & 1 & 0\\
    0 &  4 & 1 & -2 & 0 & 1\\
\end{bmatrix}
\rightarrow
\begin{bmatrix}
    1 & -2 & 0 & 1 & 0 & 0\\
    0 &  1 & 1 & 0 & 1 & 0\\
    0 &  0 & -3 & -2 & -4 & 1\\
\end{bmatrix}
\rightarrow
$$

$$
\begin{bmatrix}
    1 & -2 & 0 & 1 & 0 & 0\\ 
    0 &  1 & 0 & -2/3 & -1/3 & 1/3\\
    0 &  0 & -3 & -2 & -4 & 1\\
\end{bmatrix}
\rightarrow
\begin{bmatrix}
    1 &  0 & 0 & -1/3 & -2/3 & 2/3 \\
    0 &  1 & 0 & -2/3 & -1/3 & 1/3\\
    0 &  0 & 1 &  2/3 & 4/3  & -1/3\\
\end{bmatrix}$$

WOHO!

\section*{5}
Tanken är att skriva matrisen på övertriangulär form för att sedan multiplicera diagonalen.
\\
$R_i$ beteckna rad i.\\

$$
\begin{bmatrix}
    1  & 1   &  0   &  1 \\
    2  & 8   &  -1  &  2  \\
    4  & -4  &  0   &  3   \\
    8  &  2  &  1   &  4   \\
\end{bmatrix}
$$

$R_1 * (-2) + R_2 \rightarrow R_2$, \quad
$R_1 * (-4) + R_3 \rightarrow R_3$, \quad
$R_4 * (-8) + R_4 \rightarrow R_4$   


$$
\begin{bmatrix}
    1  & 1   &  0   &  1 \\
    0  & 6   &  -1  &  0  \\
    0  & -8  &  0   &  -1   \\
    0  &  -6  &  1   &  -4   \\
\end{bmatrix}
$$

$R_2 * (8/6) + R_3 \rightarrow R_3$, \quad
$R_2 * (1) + R_4 \rightarrow R_4$   

$$
\begin{bmatrix}
    1  & 1   &  0   &  1 \\
    0  & 6   &  -1  &  0  \\
    0  & 0   &  -4/3   &  -1   \\
    0  & 0   &  0   &  -4   \\
\end{bmatrix}
$$
$1 \cdot 6 \cdot -4/3 \cdot -4 = 32$

\section*{6}
\[ det \left| \begin{array}{ccc}
    -1 & 2 & 3 \\
     4 &-3 & 2 \\
     0 & 1 & 6 \\
\end{array} \right| = -16 \]

\[ det \left| \begin{array}{ccc}
     3 & -2  & 8 \\
     7 &  0  & 1 \\
     4 &  4  & 3 \\
\end{array} \right| = 246 \]
Det finns invers till matriserna, ty determinanten $\neq$ 0.

\end{document}
