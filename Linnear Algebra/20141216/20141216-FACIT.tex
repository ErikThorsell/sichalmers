%SI 2014-09-05

\documentclass{article}

\usepackage[T1]{fontenc}
\usepackage[utf8]{inputenc}
\usepackage[swedish]{babel}
\usepackage{fullpage}
\usepackage{amssymb}
\usepackage{bussproofs}
\usepackage{amsmath}
\usepackage{graphicx}
\usepackage{verbatim}
\usepackage{tikz}
\let\emptyset\varnothing


\title{Supplemental Instructions}
\date{
      %Place date here!
     }

\begin{document}
\maketitle


\section*{1}
\subsection*{a}
Multiplicera in 1/2 och beräkna determinanten. 
$
\\
Det(A-\lambda I) = 
\begin{vmatrix}
	1/2-\lambda & -3/2 \\
	-3/2 & 1/2-\lambda 
\end{vmatrix} 
=
(1/2-\lambda)(1/2-\lambda) - (-3/2)(-3/2)
=
\lambda^2 - \lambda - 2 = 0
$
\\
Detta ger $\lambda_1 = -1, \quad \lambda_2 = 2$ \\
$
\begin{pmatrix}
	1/2-\lambda_1 & -3/2 \\
	-3/2 & 1/2-\lambda_1 
\end{pmatrix} 
=
\begin{pmatrix}
	3/2 & -3/2 \\
	-3/2 & 3/2 
\end{pmatrix} 
=
\begin{pmatrix}
	1 & -1 \\
	0 & 0 
\end{pmatrix} 
$
\\
Därav blir 
$\mathbf{v_1} = t 
\begin{pmatrix}
	1  \\
	1
\end{pmatrix} 
$
På samma sätt kan vi räkna ut 
$\mathbf{v_2} = t 
\begin{pmatrix}
	1  \\
	-1
\end{pmatrix} 
$
\subsection*{b}
Normalisera vektorerna för att få en ON-bas och använd sedan $PDP^{-1} = PDP^t$ \\
\\
$
\frac{1}{2}
\begin{pmatrix}
	1+2^{1000} & 1-2^{1000} \\
	1-2^{1000} & 1+^{1000}
\end{pmatrix}
$
\section*{2}
Multiplicera bara vektorn med basmatrisen. \\
$F \cdot \mathbf{v}_{F}
=
\begin{pmatrix}
	9 \\
	6 \\ 
	-1
\end{pmatrix}
$

\section*{3}
$$
G = 
\begin{pmatrix}
	0 & 1 & 1 & 1 & 0 \\
	1 & 0 & 1 & 0 & 0 \\
	1 & 1 & 0 & 1 & 1 \\
	1 & 0 & 1 & 0 & 0 \\
	0 & 0 & 1 & 0 & 0
\end{pmatrix}
;\quad
M = 
\begin{pmatrix}
	0 & 1/3 & 1/3 & 1/3 & 0 \\
	1/2 & 0 & 1/2 & 0 & 0 \\
	1/4 & 1/4 & 0 & 1/4 & 1/4 \\
	1/2 & 0 & 1/2 & 0 & 0 \\
	0 & 0 & 1 & 0 & 0
\end{pmatrix}
$$

\section*{4}
Riktningsvektor $\vec{v} = (1,2,3)^t$. \\
Låt $Q$ vara den punkt på linjen som ligger närmast $P$. Vi vet då att 
$Q$ = (3+t, 4+2t, 5+3t) för något t. Vi vet även att \\

$\vec{PQ} = \vec{OQ}-\vec{OP} = 
\begin{bmatrix}
    3 + t \\
    4 + 2t \\
    5 + 3t \\
\end{bmatrix}
-
\begin{bmatrix}
    2 \\
    1 \\
    0 \\
\end{bmatrix}
=
\begin{bmatrix}
    1 + t \\
    3 + 2t \\
    5 + 3t \\
\end{bmatrix}
$
\\
$\vec{PQ}$ och $\vec{v}$ är ortogonala $\Rightarrow$ 
$0 = \vec{PQ} \cdot \vec{v}$ = 1+t + 2(3+2t) + 3(5+3t) = 22 + 14t $\Rightarrow$ 
$t = -\frac{11}{7}$.\\
Det sökta avståndet är alltså:
$
\begin{Vmatrix}
    \vec{PQ}
\end{Vmatrix}
=
\begin{Vmatrix}
    1 - 11/7 \\
    3 - 22/7 \\
    5 - 33/7 \\
\end{Vmatrix}
=
\begin{Vmatrix}
    \frac{1}{7}
        \begin{pmatrix}
            -4 \\
            -1 \\
             2 \\
        \end{pmatrix}
\end{Vmatrix}
=
\frac{1}{7} \sqrt{16+1+4} = \frac{\sqrt{21}}{7}
$

\section*{5}
Matrisen för spegling:
$
S = 
\begin{bmatrix}
    Se_{1} & Se_{2}
\end{bmatrix}
=
\begin{bmatrix}
    0 & -1 \\
    -1 & 0 \\
\end{bmatrix}
$
\\
Rotation för rotation:
$
R = 
\begin{bmatrix}
    \cos{\pi/6} & -\sin{\pi/6} \\
    \sin{\pi/6} & \cos{\pi/6} \\
\end{bmatrix}
=
\begin{bmatrix}
    \sqrt{3}/2 & -1/2 \\
    1/2 & \sqrt{3}/2
\end{bmatrix}
$
\\
$
RS = 
\frac{1}{2}
    \begin{bmatrix}
        \sqrt{3} & -1 \\
        1   & \sqrt{3} \\
    \end{bmatrix}
    \begin{bmatrix}
        0   &   -1 \\
        -1  &   0  \\
    \end{bmatrix}
=
\frac{1}{2}
    \begin{bmatrix}
        1   &   -\sqrt{3} \\
        -\sqrt{3} & -1
    \end{bmatrix}
$

\section*{6}
\subsection*{a)}
Totalmatrisen
$T = 
\begin{bmatrix}
    2  &  1  &  1  &  0 \\
   -1  &  3  & -4  &  0 \\
    a  &  2  &  1  &  0 \\
    0  &  1  & -1  &  0 \\
\end{bmatrix}
\Rightarrow
\begin{bmatrix}
   -1  &  3  & -4  &  0 \\
    0  &  7  & -7  &  0 \\
    0  &  2+3a  &  1-4a  &  0 \\
    0  &  1  & -1  &  0 \\
\end{bmatrix}
\Rightarrow
\begin{bmatrix}
   -1  &  3  & -4  &  0 \\
    0  &  1  & -1  &  0 \\
    0  &  0  &  3-a  &  0 \\
    0  &  0  &  0  &  0 \\
\end{bmatrix}
\\
$
För $a \neq 3$ har systemet entydig lösning: $x_{1} = x_{2} = x_{3} = 0$. \\
För $a = 3$ har systemet oändligt antal lösningar.

\subsection*{b)}
$\vec{v_{1}}, \vec{v_{2}}, \vec{v_{3}}$ är linjärt beroende omm ekvationssystemet i 
a) har icke-triviala lösningar; dvs: Precis för $a = 3$.

\end{document}
