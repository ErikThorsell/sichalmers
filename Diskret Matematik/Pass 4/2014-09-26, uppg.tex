%SI 2014-09-05

\documentclass{article}

\usepackage[T1]{fontenc}
\usepackage[utf8]{inputenc}
\usepackage[swedish]{babel}
\usepackage{fullpage}
\usepackage{amssymb}
\usepackage{bussproofs}

\let\emptyset\varnothing

\title{Supplemental Instructions}
\author{Erik Thorsell \\ 
		\small{erithor@student.chalmers.se}
}
\date{2014-09-26}

\begin{document}
\maketitle

\subsection*{Repetition}
Repetition är moder till all inlärning.
\begin{enumerate}

\item[1.]
Visa att det för alla positiva heltal $n$ gäller att:
$$n+3\sum\limits_{k=1}^{n} k(k-1) = n^3$$
{\it Tentamen 2004-10-22}

\item[2.]
Låt $A=\{1,2,3\}$ och $B=\{1,2\}$
\begin{itemize}
\item[a)] Hur många funktioner finns med $A$ som definitionsmängd och $B$ som målmängd?
\item[b)] Finns det någon injektiv funktion med $A$ som definitonsmängd och $B$ som målmängd?
Om någon existerar, ange en.
\item[c)] Finns det någon surjektiv funktion med $A$ som definitionsmängd och $B$ som målmängd?
Ange i så fall en sådan.
\end{itemize}
{\it Tentamen 2005-12-16}
\end{enumerate}

\subsection*{Delbarhet och diofantiska ekvationer}
\begin{enumerate}
\item[3.] 
Emil och Emilia hade varit på en lång resa. När de kom hem räknade de ut att under de dagar 
de åkt båt hade de färdats i genomsnitt 720 km/dygn. Övriga dagar hade de färdats i snitt 
400 km/dygn. Totalt hade de färdats 15 200 km. Hur många dagar hade Emil och Emilia varit på 
resande fot och hur många av dessa hade de åkt båt?\\
{\it Det var en diofantastisk resa!}\\

{\it Tentamen 2005-03-30}

\item[4.]
Beräkna $sgd(a,b)$. Avgöra sedan om det existerar en invers till [a] i $\mathbb{Z}_b$ och beräkna 
i så fall denna, då:
\begin{itemize}
\item[a)] $a = 1001$ och $b = 748$
\item[b)] $a = 317$ och $b = 70$
\item[c)] $a = 31$ och $b = 47$ 
\end{itemize}

{\it Tentamen 2005-10-21}

\item[5.]
Ge fullständig lösning till den diofantiska ekvationen $$28x+36y=100$$ och ange även inversen till $7$ i $\mathbb{Z}_9$.\\

{\it Tentamen 2005-01-11}

\item[6. ]
Beskriv Euklides utökade algoritm för att det till två heltal $a$ och $b$ finns 
$sgd(a,b)$ och två heltal $u$ och $v$ sådana att $au+bv=sgd(a,b)$.\\
Förklara varför algoritmen fungerar!\\

{\it Tentamen 2006-01-10}
\end{enumerate}

\subsection*{Primtal}
\begin{enumerate}

\item[7.]
Primtalsfaktorisera följande tal:
\begin{itemize}
\item[a) ] 1 155
\item[b) ] 340
\item[c) ] 28 675
\end{itemize}

\item[8.]
Skriv en funktion i Haskell som tar en integer $n$ som argument och returnerar en lista med intergers. 
Listan ska innehålla de $n$ första primtalen.\\

Ex:\\
ghci>>primeFunc(6)\\
$[2,3,5,7,11,13]$\\
\end{enumerate}
\end{document}
